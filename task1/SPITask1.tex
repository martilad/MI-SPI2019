
% Default to the notebook output style

    


% Inherit from the specified cell style.




    
\documentclass[11pt]{article}

    
    
    \usepackage[T1]{fontenc}
    % Nicer default font (+ math font) than Computer Modern for most use cases
    \usepackage{mathpazo}

    % Basic figure setup, for now with no caption control since it's done
    % automatically by Pandoc (which extracts ![](path) syntax from Markdown).
    \usepackage{graphicx}
    % We will generate all images so they have a width \maxwidth. This means
    % that they will get their normal width if they fit onto the page, but
    % are scaled down if they would overflow the margins.
    \makeatletter
    \def\maxwidth{\ifdim\Gin@nat@width>\linewidth\linewidth
    \else\Gin@nat@width\fi}
    \makeatother
    \let\Oldincludegraphics\includegraphics
    % Set max figure width to be 80% of text width, for now hardcoded.
    \renewcommand{\includegraphics}[1]{\Oldincludegraphics[width=.8\maxwidth]{#1}}
    % Ensure that by default, figures have no caption (until we provide a
    % proper Figure object with a Caption API and a way to capture that
    % in the conversion process - todo).
    \usepackage{caption}
    \DeclareCaptionLabelFormat{nolabel}{}
    \captionsetup{labelformat=nolabel}

    \usepackage{adjustbox} % Used to constrain images to a maximum size 
    \usepackage{xcolor} % Allow colors to be defined
    \usepackage{enumerate} % Needed for markdown enumerations to work
    \usepackage{geometry} % Used to adjust the document margins
    \usepackage{amsmath} % Equations
    \usepackage{amssymb} % Equations
    \usepackage{textcomp} % defines textquotesingle
    % Hack from http://tex.stackexchange.com/a/47451/13684:
    \AtBeginDocument{%
        \def\PYZsq{\textquotesingle}% Upright quotes in Pygmentized code
    }
    \usepackage{upquote} % Upright quotes for verbatim code
    \usepackage{eurosym} % defines \euro
    \usepackage[mathletters]{ucs} % Extended unicode (utf-8) support
    \usepackage[utf8x]{inputenc} % Allow utf-8 characters in the tex document
    \usepackage{fancyvrb} % verbatim replacement that allows latex
    \usepackage{grffile} % extends the file name processing of package graphics 
                         % to support a larger range 
    % The hyperref package gives us a pdf with properly built
    % internal navigation ('pdf bookmarks' for the table of contents,
    % internal cross-reference links, web links for URLs, etc.)
    \usepackage{hyperref}
    \usepackage{longtable} % longtable support required by pandoc >1.10
    \usepackage{booktabs}  % table support for pandoc > 1.12.2
    \usepackage[inline]{enumitem} % IRkernel/repr support (it uses the enumerate* environment)
    \usepackage[normalem]{ulem} % ulem is needed to support strikethroughs (\sout)
                                % normalem makes italics be italics, not underlines
    \usepackage{mathrsfs}
    

    
    
    % Colors for the hyperref package
    \definecolor{urlcolor}{rgb}{0,.145,.698}
    \definecolor{linkcolor}{rgb}{.71,0.21,0.01}
    \definecolor{citecolor}{rgb}{.12,.54,.11}

    % ANSI colors
    \definecolor{ansi-black}{HTML}{3E424D}
    \definecolor{ansi-black-intense}{HTML}{282C36}
    \definecolor{ansi-red}{HTML}{E75C58}
    \definecolor{ansi-red-intense}{HTML}{B22B31}
    \definecolor{ansi-green}{HTML}{00A250}
    \definecolor{ansi-green-intense}{HTML}{007427}
    \definecolor{ansi-yellow}{HTML}{DDB62B}
    \definecolor{ansi-yellow-intense}{HTML}{B27D12}
    \definecolor{ansi-blue}{HTML}{208FFB}
    \definecolor{ansi-blue-intense}{HTML}{0065CA}
    \definecolor{ansi-magenta}{HTML}{D160C4}
    \definecolor{ansi-magenta-intense}{HTML}{A03196}
    \definecolor{ansi-cyan}{HTML}{60C6C8}
    \definecolor{ansi-cyan-intense}{HTML}{258F8F}
    \definecolor{ansi-white}{HTML}{C5C1B4}
    \definecolor{ansi-white-intense}{HTML}{A1A6B2}
    \definecolor{ansi-default-inverse-fg}{HTML}{FFFFFF}
    \definecolor{ansi-default-inverse-bg}{HTML}{000000}

    % commands and environments needed by pandoc snippets
    % extracted from the output of `pandoc -s`
    \providecommand{\tightlist}{%
      \setlength{\itemsep}{0pt}\setlength{\parskip}{0pt}}
    \DefineVerbatimEnvironment{Highlighting}{Verbatim}{commandchars=\\\{\}}
    % Add ',fontsize=\small' for more characters per line
    \newenvironment{Shaded}{}{}
    \newcommand{\KeywordTok}[1]{\textcolor[rgb]{0.00,0.44,0.13}{\textbf{{#1}}}}
    \newcommand{\DataTypeTok}[1]{\textcolor[rgb]{0.56,0.13,0.00}{{#1}}}
    \newcommand{\DecValTok}[1]{\textcolor[rgb]{0.25,0.63,0.44}{{#1}}}
    \newcommand{\BaseNTok}[1]{\textcolor[rgb]{0.25,0.63,0.44}{{#1}}}
    \newcommand{\FloatTok}[1]{\textcolor[rgb]{0.25,0.63,0.44}{{#1}}}
    \newcommand{\CharTok}[1]{\textcolor[rgb]{0.25,0.44,0.63}{{#1}}}
    \newcommand{\StringTok}[1]{\textcolor[rgb]{0.25,0.44,0.63}{{#1}}}
    \newcommand{\CommentTok}[1]{\textcolor[rgb]{0.38,0.63,0.69}{\textit{{#1}}}}
    \newcommand{\OtherTok}[1]{\textcolor[rgb]{0.00,0.44,0.13}{{#1}}}
    \newcommand{\AlertTok}[1]{\textcolor[rgb]{1.00,0.00,0.00}{\textbf{{#1}}}}
    \newcommand{\FunctionTok}[1]{\textcolor[rgb]{0.02,0.16,0.49}{{#1}}}
    \newcommand{\RegionMarkerTok}[1]{{#1}}
    \newcommand{\ErrorTok}[1]{\textcolor[rgb]{1.00,0.00,0.00}{\textbf{{#1}}}}
    \newcommand{\NormalTok}[1]{{#1}}
    
    % Additional commands for more recent versions of Pandoc
    \newcommand{\ConstantTok}[1]{\textcolor[rgb]{0.53,0.00,0.00}{{#1}}}
    \newcommand{\SpecialCharTok}[1]{\textcolor[rgb]{0.25,0.44,0.63}{{#1}}}
    \newcommand{\VerbatimStringTok}[1]{\textcolor[rgb]{0.25,0.44,0.63}{{#1}}}
    \newcommand{\SpecialStringTok}[1]{\textcolor[rgb]{0.73,0.40,0.53}{{#1}}}
    \newcommand{\ImportTok}[1]{{#1}}
    \newcommand{\DocumentationTok}[1]{\textcolor[rgb]{0.73,0.13,0.13}{\textit{{#1}}}}
    \newcommand{\AnnotationTok}[1]{\textcolor[rgb]{0.38,0.63,0.69}{\textbf{\textit{{#1}}}}}
    \newcommand{\CommentVarTok}[1]{\textcolor[rgb]{0.38,0.63,0.69}{\textbf{\textit{{#1}}}}}
    \newcommand{\VariableTok}[1]{\textcolor[rgb]{0.10,0.09,0.49}{{#1}}}
    \newcommand{\ControlFlowTok}[1]{\textcolor[rgb]{0.00,0.44,0.13}{\textbf{{#1}}}}
    \newcommand{\OperatorTok}[1]{\textcolor[rgb]{0.40,0.40,0.40}{{#1}}}
    \newcommand{\BuiltInTok}[1]{{#1}}
    \newcommand{\ExtensionTok}[1]{{#1}}
    \newcommand{\PreprocessorTok}[1]{\textcolor[rgb]{0.74,0.48,0.00}{{#1}}}
    \newcommand{\AttributeTok}[1]{\textcolor[rgb]{0.49,0.56,0.16}{{#1}}}
    \newcommand{\InformationTok}[1]{\textcolor[rgb]{0.38,0.63,0.69}{\textbf{\textit{{#1}}}}}
    \newcommand{\WarningTok}[1]{\textcolor[rgb]{0.38,0.63,0.69}{\textbf{\textit{{#1}}}}}
    
    
    % Define a nice break command that doesn't care if a line doesn't already
    % exist.
    \def\br{\hspace*{\fill} \\* }
    % Math Jax compatibility definitions
    \def\gt{>}
    \def\lt{<}
    \let\Oldtex\TeX
    \let\Oldlatex\LaTeX
    \renewcommand{\TeX}{\textrm{\Oldtex}}
    \renewcommand{\LaTeX}{\textrm{\Oldlatex}}
    % Document parameters
    % Document title
    \title{Domácí úkol 1}
    \author{Ladislav Martínek a Richard Werner}
    
    
    
    
    

    % Pygments definitions
    
\makeatletter
\def\PY@reset{\let\PY@it=\relax \let\PY@bf=\relax%
    \let\PY@ul=\relax \let\PY@tc=\relax%
    \let\PY@bc=\relax \let\PY@ff=\relax}
\def\PY@tok#1{\csname PY@tok@#1\endcsname}
\def\PY@toks#1+{\ifx\relax#1\empty\else%
    \PY@tok{#1}\expandafter\PY@toks\fi}
\def\PY@do#1{\PY@bc{\PY@tc{\PY@ul{%
    \PY@it{\PY@bf{\PY@ff{#1}}}}}}}
\def\PY#1#2{\PY@reset\PY@toks#1+\relax+\PY@do{#2}}

\expandafter\def\csname PY@tok@mi\endcsname{\def\PY@tc##1{\textcolor[rgb]{0.40,0.40,0.40}{##1}}}
\expandafter\def\csname PY@tok@gh\endcsname{\let\PY@bf=\textbf\def\PY@tc##1{\textcolor[rgb]{0.00,0.00,0.50}{##1}}}
\expandafter\def\csname PY@tok@se\endcsname{\let\PY@bf=\textbf\def\PY@tc##1{\textcolor[rgb]{0.73,0.40,0.13}{##1}}}
\expandafter\def\csname PY@tok@go\endcsname{\def\PY@tc##1{\textcolor[rgb]{0.53,0.53,0.53}{##1}}}
\expandafter\def\csname PY@tok@no\endcsname{\def\PY@tc##1{\textcolor[rgb]{0.53,0.00,0.00}{##1}}}
\expandafter\def\csname PY@tok@vg\endcsname{\def\PY@tc##1{\textcolor[rgb]{0.10,0.09,0.49}{##1}}}
\expandafter\def\csname PY@tok@nb\endcsname{\def\PY@tc##1{\textcolor[rgb]{0.00,0.50,0.00}{##1}}}
\expandafter\def\csname PY@tok@gt\endcsname{\def\PY@tc##1{\textcolor[rgb]{0.00,0.27,0.87}{##1}}}
\expandafter\def\csname PY@tok@vi\endcsname{\def\PY@tc##1{\textcolor[rgb]{0.10,0.09,0.49}{##1}}}
\expandafter\def\csname PY@tok@mb\endcsname{\def\PY@tc##1{\textcolor[rgb]{0.40,0.40,0.40}{##1}}}
\expandafter\def\csname PY@tok@gr\endcsname{\def\PY@tc##1{\textcolor[rgb]{1.00,0.00,0.00}{##1}}}
\expandafter\def\csname PY@tok@gs\endcsname{\let\PY@bf=\textbf}
\expandafter\def\csname PY@tok@cp\endcsname{\def\PY@tc##1{\textcolor[rgb]{0.74,0.48,0.00}{##1}}}
\expandafter\def\csname PY@tok@nc\endcsname{\let\PY@bf=\textbf\def\PY@tc##1{\textcolor[rgb]{0.00,0.00,1.00}{##1}}}
\expandafter\def\csname PY@tok@kn\endcsname{\let\PY@bf=\textbf\def\PY@tc##1{\textcolor[rgb]{0.00,0.50,0.00}{##1}}}
\expandafter\def\csname PY@tok@sr\endcsname{\def\PY@tc##1{\textcolor[rgb]{0.73,0.40,0.53}{##1}}}
\expandafter\def\csname PY@tok@sd\endcsname{\let\PY@it=\textit\def\PY@tc##1{\textcolor[rgb]{0.73,0.13,0.13}{##1}}}
\expandafter\def\csname PY@tok@ne\endcsname{\let\PY@bf=\textbf\def\PY@tc##1{\textcolor[rgb]{0.82,0.25,0.23}{##1}}}
\expandafter\def\csname PY@tok@kr\endcsname{\let\PY@bf=\textbf\def\PY@tc##1{\textcolor[rgb]{0.00,0.50,0.00}{##1}}}
\expandafter\def\csname PY@tok@sc\endcsname{\def\PY@tc##1{\textcolor[rgb]{0.73,0.13,0.13}{##1}}}
\expandafter\def\csname PY@tok@ss\endcsname{\def\PY@tc##1{\textcolor[rgb]{0.10,0.09,0.49}{##1}}}
\expandafter\def\csname PY@tok@bp\endcsname{\def\PY@tc##1{\textcolor[rgb]{0.00,0.50,0.00}{##1}}}
\expandafter\def\csname PY@tok@gp\endcsname{\let\PY@bf=\textbf\def\PY@tc##1{\textcolor[rgb]{0.00,0.00,0.50}{##1}}}
\expandafter\def\csname PY@tok@kc\endcsname{\let\PY@bf=\textbf\def\PY@tc##1{\textcolor[rgb]{0.00,0.50,0.00}{##1}}}
\expandafter\def\csname PY@tok@c1\endcsname{\let\PY@it=\textit\def\PY@tc##1{\textcolor[rgb]{0.25,0.50,0.50}{##1}}}
\expandafter\def\csname PY@tok@fm\endcsname{\def\PY@tc##1{\textcolor[rgb]{0.00,0.00,1.00}{##1}}}
\expandafter\def\csname PY@tok@gu\endcsname{\let\PY@bf=\textbf\def\PY@tc##1{\textcolor[rgb]{0.50,0.00,0.50}{##1}}}
\expandafter\def\csname PY@tok@nt\endcsname{\let\PY@bf=\textbf\def\PY@tc##1{\textcolor[rgb]{0.00,0.50,0.00}{##1}}}
\expandafter\def\csname PY@tok@s\endcsname{\def\PY@tc##1{\textcolor[rgb]{0.73,0.13,0.13}{##1}}}
\expandafter\def\csname PY@tok@sx\endcsname{\def\PY@tc##1{\textcolor[rgb]{0.00,0.50,0.00}{##1}}}
\expandafter\def\csname PY@tok@kt\endcsname{\def\PY@tc##1{\textcolor[rgb]{0.69,0.00,0.25}{##1}}}
\expandafter\def\csname PY@tok@err\endcsname{\def\PY@bc##1{\setlength{\fboxsep}{0pt}\fcolorbox[rgb]{1.00,0.00,0.00}{1,1,1}{\strut ##1}}}
\expandafter\def\csname PY@tok@kd\endcsname{\let\PY@bf=\textbf\def\PY@tc##1{\textcolor[rgb]{0.00,0.50,0.00}{##1}}}
\expandafter\def\csname PY@tok@w\endcsname{\def\PY@tc##1{\textcolor[rgb]{0.73,0.73,0.73}{##1}}}
\expandafter\def\csname PY@tok@kp\endcsname{\def\PY@tc##1{\textcolor[rgb]{0.00,0.50,0.00}{##1}}}
\expandafter\def\csname PY@tok@dl\endcsname{\def\PY@tc##1{\textcolor[rgb]{0.73,0.13,0.13}{##1}}}
\expandafter\def\csname PY@tok@sa\endcsname{\def\PY@tc##1{\textcolor[rgb]{0.73,0.13,0.13}{##1}}}
\expandafter\def\csname PY@tok@k\endcsname{\let\PY@bf=\textbf\def\PY@tc##1{\textcolor[rgb]{0.00,0.50,0.00}{##1}}}
\expandafter\def\csname PY@tok@o\endcsname{\def\PY@tc##1{\textcolor[rgb]{0.40,0.40,0.40}{##1}}}
\expandafter\def\csname PY@tok@il\endcsname{\def\PY@tc##1{\textcolor[rgb]{0.40,0.40,0.40}{##1}}}
\expandafter\def\csname PY@tok@cpf\endcsname{\let\PY@it=\textit\def\PY@tc##1{\textcolor[rgb]{0.25,0.50,0.50}{##1}}}
\expandafter\def\csname PY@tok@nf\endcsname{\def\PY@tc##1{\textcolor[rgb]{0.00,0.00,1.00}{##1}}}
\expandafter\def\csname PY@tok@gi\endcsname{\def\PY@tc##1{\textcolor[rgb]{0.00,0.63,0.00}{##1}}}
\expandafter\def\csname PY@tok@nv\endcsname{\def\PY@tc##1{\textcolor[rgb]{0.10,0.09,0.49}{##1}}}
\expandafter\def\csname PY@tok@ow\endcsname{\let\PY@bf=\textbf\def\PY@tc##1{\textcolor[rgb]{0.67,0.13,1.00}{##1}}}
\expandafter\def\csname PY@tok@mf\endcsname{\def\PY@tc##1{\textcolor[rgb]{0.40,0.40,0.40}{##1}}}
\expandafter\def\csname PY@tok@vc\endcsname{\def\PY@tc##1{\textcolor[rgb]{0.10,0.09,0.49}{##1}}}
\expandafter\def\csname PY@tok@na\endcsname{\def\PY@tc##1{\textcolor[rgb]{0.49,0.56,0.16}{##1}}}
\expandafter\def\csname PY@tok@ni\endcsname{\let\PY@bf=\textbf\def\PY@tc##1{\textcolor[rgb]{0.60,0.60,0.60}{##1}}}
\expandafter\def\csname PY@tok@ch\endcsname{\let\PY@it=\textit\def\PY@tc##1{\textcolor[rgb]{0.25,0.50,0.50}{##1}}}
\expandafter\def\csname PY@tok@s1\endcsname{\def\PY@tc##1{\textcolor[rgb]{0.73,0.13,0.13}{##1}}}
\expandafter\def\csname PY@tok@mo\endcsname{\def\PY@tc##1{\textcolor[rgb]{0.40,0.40,0.40}{##1}}}
\expandafter\def\csname PY@tok@m\endcsname{\def\PY@tc##1{\textcolor[rgb]{0.40,0.40,0.40}{##1}}}
\expandafter\def\csname PY@tok@nd\endcsname{\def\PY@tc##1{\textcolor[rgb]{0.67,0.13,1.00}{##1}}}
\expandafter\def\csname PY@tok@ge\endcsname{\let\PY@it=\textit}
\expandafter\def\csname PY@tok@mh\endcsname{\def\PY@tc##1{\textcolor[rgb]{0.40,0.40,0.40}{##1}}}
\expandafter\def\csname PY@tok@cm\endcsname{\let\PY@it=\textit\def\PY@tc##1{\textcolor[rgb]{0.25,0.50,0.50}{##1}}}
\expandafter\def\csname PY@tok@s2\endcsname{\def\PY@tc##1{\textcolor[rgb]{0.73,0.13,0.13}{##1}}}
\expandafter\def\csname PY@tok@c\endcsname{\let\PY@it=\textit\def\PY@tc##1{\textcolor[rgb]{0.25,0.50,0.50}{##1}}}
\expandafter\def\csname PY@tok@cs\endcsname{\let\PY@it=\textit\def\PY@tc##1{\textcolor[rgb]{0.25,0.50,0.50}{##1}}}
\expandafter\def\csname PY@tok@si\endcsname{\let\PY@bf=\textbf\def\PY@tc##1{\textcolor[rgb]{0.73,0.40,0.53}{##1}}}
\expandafter\def\csname PY@tok@sb\endcsname{\def\PY@tc##1{\textcolor[rgb]{0.73,0.13,0.13}{##1}}}
\expandafter\def\csname PY@tok@vm\endcsname{\def\PY@tc##1{\textcolor[rgb]{0.10,0.09,0.49}{##1}}}
\expandafter\def\csname PY@tok@nl\endcsname{\def\PY@tc##1{\textcolor[rgb]{0.63,0.63,0.00}{##1}}}
\expandafter\def\csname PY@tok@sh\endcsname{\def\PY@tc##1{\textcolor[rgb]{0.73,0.13,0.13}{##1}}}
\expandafter\def\csname PY@tok@nn\endcsname{\let\PY@bf=\textbf\def\PY@tc##1{\textcolor[rgb]{0.00,0.00,1.00}{##1}}}
\expandafter\def\csname PY@tok@gd\endcsname{\def\PY@tc##1{\textcolor[rgb]{0.63,0.00,0.00}{##1}}}

\def\PYZbs{\char`\\}
\def\PYZus{\char`\_}
\def\PYZob{\char`\{}
\def\PYZcb{\char`\}}
\def\PYZca{\char`\^}
\def\PYZam{\char`\&}
\def\PYZlt{\char`\<}
\def\PYZgt{\char`\>}
\def\PYZsh{\char`\#}
\def\PYZpc{\char`\%}
\def\PYZdl{\char`\$}
\def\PYZhy{\char`\-}
\def\PYZsq{\char`\'}
\def\PYZdq{\char`\"}
\def\PYZti{\char`\~}
% for compatibility with earlier versions
\def\PYZat{@}
\def\PYZlb{[}
\def\PYZrb{]}
\makeatother


    % Exact colors from NB
    \definecolor{incolor}{rgb}{0.0, 0.0, 0.5}
    \definecolor{outcolor}{rgb}{0.545, 0.0, 0.0}



    
    % Prevent overflowing lines due to hard-to-break entities
    \sloppy 
    % Setup hyperref package
    \hypersetup{
      breaklinks=true,  % so long urls are correctly broken across lines
      colorlinks=true,
      urlcolor=urlcolor,
      linkcolor=linkcolor,
      citecolor=citecolor,
      }
    % Slightly bigger margins than the latex defaults
    
    \geometry{verbose,tmargin=1in,bmargin=1in,lmargin=1in,rmargin=1in}
    
    

    \begin{document}
    
    
    \maketitle
    
    

    
    \hypertarget{domuxe1cuxed-uxfakol-1-6-bodux16f}{%
\section{Domácí úkol 1 (6
bodů)}\label{domuxe1cuxed-uxfakol-1-6-bodux16f}}

    \hypertarget{uxfakoly}{%
\subsection{Úkoly}\label{uxfakoly}}

\begin{enumerate}
\def\labelenumi{\arabic{enumi}.}
\tightlist
\item
  (1b) Z obou datových souborů načtěte texty k analýze. Pro každý text
  zvlášť odhadněte základní charakteristiky délek slov, tj. střední
  hodnotu a rozptyl. Graficky znázorněte rozdělení délek slov.
\item
  (1b) Pro každý text zvlášť odhadněte pravděpodobnosti písmen (symbolů
  mimo mezery), které se v textech vyskytují. Výsledné pravděpodobnosti
  graficky znázorněte.
\item
  (1.5b) Na hladině významnosti 5\% otestujte hypotézu, že rozdělení
  délek slov nezávisí na tom, o který jde text. Určete také p-hodnotu
  testu.
\item
  (1.5b) Na hladině významnosti 5\% otestujte hypotézu, že se střední
  délky slov v obou textech rovnají. Určete také p-hodnotu testu.
\item
  (1b) Na hladině významnosti 5\% otestujte hypotézu, že rozdělení
  písmen nezávisí na tom, o který jde text. Určete také p-hodnotu testu.
\end{enumerate}
\newpage
    \hypertarget{ux159eux161enuxed}{%
\subsection{Řešení}\label{ux159eux161enuxed}}

    \begin{Verbatim}[commandchars=\\\{\}]
{\color{incolor}In [{\color{incolor}1}]:} \PY{k+kn}{import} \PY{n+nn}{numpy} \PY{k}{as} \PY{n+nn}{np}
        \PY{k+kn}{import} \PY{n+nn}{pandas} \PY{k}{as} \PY{n+nn}{pd}
        \PY{k+kn}{import} \PY{n+nn}{matplotlib}\PY{n+nn}{.}\PY{n+nn}{pyplot} \PY{k}{as} \PY{n+nn}{plt}
        \PY{k+kn}{from} \PY{n+nn}{scipy} \PY{k}{import} \PY{n}{stats}
        \PY{k+kn}{from} \PY{n+nn}{scipy}\PY{n+nn}{.}\PY{n+nn}{optimize} \PY{k}{import} \PY{n}{minimize}
\end{Verbatim}

    \hypertarget{zpracovuxe1nuxed-souborux16f}{%
\subsubsection{Zpracování souborů}\label{zpracovuxe1nuxed-souborux16f}}

    \begin{Verbatim}[commandchars=\\\{\}]
{\color{incolor}In [{\color{incolor}2}]:} \PY{n}{K} \PY{o}{=} \PY{l+m+mi}{15}
        \PY{n}{L} \PY{o}{=} \PY{n+nb}{len}\PY{p}{(}\PY{l+s+s2}{\PYZdq{}}\PY{l+s+s2}{Martínek}\PY{l+s+s2}{\PYZdq{}}\PY{p}{)}
        \PY{n}{X} \PY{o}{=} \PY{p}{(}\PY{p}{(}\PY{n}{K}\PY{o}{*}\PY{n}{L}\PY{o}{*}\PY{l+m+mi}{23}\PY{p}{)} \PY{o}{\PYZpc{}} \PY{p}{(}\PY{l+m+mi}{20}\PY{p}{)}\PY{p}{)} \PY{o}{+} \PY{l+m+mi}{1}
        \PY{n}{X\PYZus{}file} \PY{o}{=} \PY{l+s+s1}{\PYZsq{}}\PY{l+s+s1}{0}\PY{l+s+s1}{\PYZsq{}}\PY{o}{*}\PY{p}{(}\PY{l+m+mi}{3}\PY{o}{\PYZhy{}}\PY{n+nb}{len}\PY{p}{(}\PY{n+nb}{str}\PY{p}{(}\PY{n}{X}\PY{p}{)}\PY{p}{)}\PY{p}{)}\PY{o}{+}\PY{n+nb}{str}\PY{p}{(}\PY{n}{X}\PY{p}{)}\PY{o}{+}\PY{l+s+s1}{\PYZsq{}}\PY{l+s+s1}{.txt}\PY{l+s+s1}{\PYZsq{}}
        \PY{n}{Y} \PY{o}{=} \PY{p}{(}\PY{p}{(}\PY{n}{X} \PY{o}{+} \PY{p}{(}\PY{p}{(}\PY{n}{K}\PY{o}{*}\PY{l+m+mi}{5} \PY{o}{+} \PY{n}{L}\PY{o}{*}\PY{l+m+mi}{7}\PY{p}{)} \PY{o}{\PYZpc{}} \PY{p}{(}\PY{l+m+mi}{19}\PY{p}{)}\PY{p}{)}\PY{p}{)} \PY{o}{\PYZpc{}} \PY{p}{(}\PY{l+m+mi}{20}\PY{p}{)}\PY{p}{)} \PY{o}{+} \PY{l+m+mi}{1}
        \PY{n}{Y\PYZus{}file} \PY{o}{=} \PY{l+s+s1}{\PYZsq{}}\PY{l+s+s1}{0}\PY{l+s+s1}{\PYZsq{}}\PY{o}{*}\PY{p}{(}\PY{l+m+mi}{3}\PY{o}{\PYZhy{}}\PY{n+nb}{len}\PY{p}{(}\PY{n+nb}{str}\PY{p}{(}\PY{n}{Y}\PY{p}{)}\PY{p}{)}\PY{p}{)}\PY{o}{+}\PY{n+nb}{str}\PY{p}{(}\PY{n}{Y}\PY{p}{)}\PY{o}{+}\PY{l+s+s1}{\PYZsq{}}\PY{l+s+s1}{.txt}\PY{l+s+s1}{\PYZsq{}}
\end{Verbatim}

    \begin{Verbatim}[commandchars=\\\{\}]
{\color{incolor}In [{\color{incolor}3}]:} \PY{k}{def} \PY{n+nf}{read\PYZus{}whole\PYZus{}file}\PY{p}{(}\PY{n}{filename}\PY{p}{)}\PY{p}{:}
            \PY{k}{with} \PY{n+nb}{open}\PY{p}{(}\PY{n}{filename}\PY{p}{,} \PY{l+s+s1}{\PYZsq{}}\PY{l+s+s1}{r}\PY{l+s+s1}{\PYZsq{}}\PY{p}{)} \PY{k}{as} \PY{n}{file}\PY{p}{:}
                \PY{k}{return} \PY{n}{file}\PY{o}{.}\PY{n}{read}\PY{p}{(}\PY{p}{)}
        
        \PY{n}{xfile} \PY{o}{=} \PY{n}{read\PYZus{}whole\PYZus{}file}\PY{p}{(}\PY{n}{X\PYZus{}file}\PY{p}{)}
        \PY{n}{yfile} \PY{o}{=} \PY{n}{read\PYZus{}whole\PYZus{}file}\PY{p}{(}\PY{n}{Y\PYZus{}file}\PY{p}{)}
\end{Verbatim}

    \hypertarget{pux159uxedklad-1}{%
\subsubsection{Příklad 1}\label{pux159uxedklad-1}}

\hypertarget{vuxfdpoux10det-vuxfdbux11brovuxe9-stux159ednuxed-hodnoty-a-vuxfdbux11brovuxe9ho-rozptylu}{%
\paragraph{Výpočet výběrové střední hodnoty a výběrového
rozptylu}\label{vuxfdpoux10det-vuxfdbux11brovuxe9-stux159ednuxed-hodnoty-a-vuxfdbux11brovuxe9ho-rozptylu}}

\begin{itemize}
\tightlist
\item
  U rozptylu bylo nutné jako parametr předat hodnotu 1 jako odečtení
  stupně volnosti, aby byl rozptyl nestranný.
\end{itemize}

\hypertarget{grafickuxe9-znuxe1zornux11bnuxed-ux10detnostuxed}{%
\paragraph{Grafické znázornění
četností}\label{grafickuxe9-znuxe1zornux11bnuxed-ux10detnostuxed}}

\begin{itemize}
\tightlist
\item
  Vytvořili jsme grafy jak pro každý soubor zvlášť, tak pro soubory jako
  korpus.
\item
  Jelikož v zadání byla řečena délka slov, vyfiltrovali jsme tečky a
  čárky přilepené ke slovům.
\end{itemize}

    \begin{Verbatim}[commandchars=\\\{\}]
{\color{incolor}In [{\color{incolor}4}]:} \PY{k}{def} \PY{n+nf}{get\PYZus{}word\PYZus{}lengths}\PY{p}{(}\PY{n}{file\PYZus{}str}\PY{p}{)}\PY{p}{:}
            \PY{n}{word\PYZus{}list} \PY{o}{=} \PY{n}{file\PYZus{}str}\PY{o}{.}\PY{n}{split}\PY{p}{(}\PY{p}{)}
            \PY{k}{return} \PY{p}{[}\PY{n+nb}{len}\PY{p}{(}\PY{n}{word}\PY{o}{.}\PY{n}{replace}\PY{p}{(}\PY{l+s+s1}{\PYZsq{}}\PY{l+s+s1}{,}\PY{l+s+s1}{\PYZsq{}}\PY{p}{,} \PY{l+s+s1}{\PYZsq{}}\PY{l+s+s1}{\PYZsq{}}\PY{p}{)}\PY{o}{.}\PY{n}{replace}\PY{p}{(}\PY{l+s+s1}{\PYZsq{}}\PY{l+s+s1}{.}\PY{l+s+s1}{\PYZsq{}}\PY{p}{,} \PY{l+s+s1}{\PYZsq{}}\PY{l+s+s1}{\PYZsq{}}\PY{p}{)}\PY{p}{)} \PY{k}{for} \PY{n}{word} \PY{o+ow}{in} \PY{n}{word\PYZus{}list}\PY{p}{]}
\end{Verbatim}

    \begin{Verbatim}[commandchars=\\\{\}]
{\color{incolor}In [{\color{incolor}5}]:} \PY{n}{Xlengths} \PY{o}{=} \PY{n}{get\PYZus{}word\PYZus{}lengths}\PY{p}{(}\PY{n}{xfile}\PY{p}{)}
        \PY{n}{Ylengths} \PY{o}{=} \PY{n}{get\PYZus{}word\PYZus{}lengths}\PY{p}{(}\PY{n}{yfile}\PY{p}{)}
        
        \PY{k}{for} \PY{n}{lengths}\PY{p}{,} \PY{n}{name} \PY{o+ow}{in} \PY{p}{[}\PY{p}{(}\PY{n}{Xlengths}\PY{p}{,} \PY{l+s+s1}{\PYZsq{}}\PY{l+s+s1}{X}\PY{l+s+s1}{\PYZsq{}}\PY{p}{)}\PY{p}{,} \PY{p}{(}\PY{n}{Ylengths}\PY{p}{,} \PY{l+s+s1}{\PYZsq{}}\PY{l+s+s1}{Y}\PY{l+s+s1}{\PYZsq{}}\PY{p}{)}\PY{p}{]}\PY{p}{:}
            \PY{n+nb}{print}\PY{p}{(}\PY{l+s+s2}{\PYZdq{}}\PY{l+s+s2}{Soubor }\PY{l+s+si}{\PYZob{}\PYZcb{}}\PY{l+s+se}{\PYZbs{}n}\PY{l+s+s2}{Výběrový průměr: }\PY{l+s+si}{\PYZob{}\PYZcb{}}\PY{l+s+se}{\PYZbs{}n}\PY{l+s+s2}{Výběrový rozptyl: }\PY{l+s+si}{\PYZob{}\PYZcb{}}\PY{l+s+se}{\PYZbs{}n}\PY{l+s+s2}{\PYZdq{}}
                  \PY{o}{.}\PY{n}{format}\PY{p}{(}\PY{n}{name}\PY{p}{,} \PY{n}{np}\PY{o}{.}\PY{n}{mean}\PY{p}{(}\PY{n}{lengths}\PY{p}{)}\PY{p}{,} \PY{n}{np}\PY{o}{.}\PY{n}{var}\PY{p}{(}\PY{n}{lengths}\PY{p}{,} \PY{n}{ddof}\PY{o}{=}\PY{l+m+mi}{1}\PY{p}{)}\PY{p}{)}\PY{p}{)}
\end{Verbatim}
\newpage
    \begin{Verbatim}[commandchars=\\\{\}]
Soubor X
Výběrový průměr: 4.379966887417218
Výběrový rozptyl: 4.1463091952572455

Soubor Y
Výběrový průměr: 4.6476595744680855
Výběrový rozptyl: 6.8297538874188986


    \end{Verbatim}

    \begin{Verbatim}[commandchars=\\\{\}]
{\color{incolor}In [{\color{incolor}6}]:} \PY{n}{x\PYZus{}length\PYZus{}set}\PY{p}{,} \PY{n}{x\PYZus{}length\PYZus{}counts} \PY{o}{=} \PY{n}{np}\PY{o}{.}\PY{n}{unique}\PY{p}{(}\PY{n}{Xlengths}\PY{p}{,} \PY{n}{return\PYZus{}counts}\PY{o}{=}\PY{k+kc}{True}\PY{p}{)}
        \PY{n}{y\PYZus{}length\PYZus{}set}\PY{p}{,} \PY{n}{y\PYZus{}length\PYZus{}counts} \PY{o}{=} \PY{n}{np}\PY{o}{.}\PY{n}{unique}\PY{p}{(}\PY{n}{Ylengths}\PY{p}{,} \PY{n}{return\PYZus{}counts}\PY{o}{=}\PY{k+kc}{True}\PY{p}{)}
\end{Verbatim}



    \begin{center}
    \adjustimage{max size={0.9\linewidth}{0.9\paperheight}}{output_11_0.png}
    \end{center}
    { \hspace*{\fill} \\}
    

    \begin{center}
    \adjustimage{max size={0.9\linewidth}{0.9\paperheight}}{output_12_0.png}
    \end{center}
    { \hspace*{\fill} \\}
    \newpage
    \hypertarget{pux159uxedklad-2}{%
\subsubsection{Příklad 2}\label{pux159uxedklad-2}}

\hypertarget{grafickuxe9-znuxe1zornux11bnuxed-pravdux11bpodobnostuxed-symbolux16f}{%
\paragraph{Grafické znázornění pravděpodobností
symbolů}\label{grafickuxe9-znuxe1zornux11bnuxed-pravdux11bpodobnostuxed-symbolux16f}}

\begin{itemize}
\tightlist
\item
  Jelikož je v zadání řečeno ``symboly mimo mezery'', do statistik jsme
  započítali i znaky pro tečky a čárky.
\end{itemize}

    \begin{Verbatim}[commandchars=\\\{\}]
{\color{incolor}In [{\color{incolor}9}]:} \PY{n}{xletters}\PY{p}{,} \PY{n}{xletter\PYZus{}cnt} \PY{o}{=} \PY{n}{np}\PY{o}{.}\PY{n}{unique}\PY{p}{(}\PY{n+nb}{list}\PY{p}{(}\PY{n}{xfile}\PY{o}{.}\PY{n}{replace}\PY{p}{(}\PY{l+s+s2}{\PYZdq{}}\PY{l+s+s2}{ }\PY{l+s+s2}{\PYZdq{}}\PY{p}{,} \PY{l+s+s2}{\PYZdq{}}\PY{l+s+s2}{\PYZdq{}}\PY{p}{)}
                                               \PY{o}{.}\PY{n}{replace}\PY{p}{(}\PY{l+s+s2}{\PYZdq{}}\PY{l+s+se}{\PYZbs{}n}\PY{l+s+s2}{\PYZdq{}}\PY{p}{,} \PY{l+s+s2}{\PYZdq{}}\PY{l+s+s2}{\PYZdq{}}\PY{p}{)}\PY{o}{.}\PY{n}{lower}\PY{p}{(}\PY{p}{)}\PY{p}{)}\PY{p}{,} 
                                          \PY{n}{return\PYZus{}counts}\PY{o}{=}\PY{k+kc}{True}\PY{p}{)}
        \PY{n}{yletters}\PY{p}{,} \PY{n}{yletter\PYZus{}cnt} \PY{o}{=} \PY{n}{np}\PY{o}{.}\PY{n}{unique}\PY{p}{(}\PY{n+nb}{list}\PY{p}{(}\PY{n}{yfile}\PY{o}{.}\PY{n}{replace}\PY{p}{(}\PY{l+s+s2}{\PYZdq{}}\PY{l+s+s2}{ }\PY{l+s+s2}{\PYZdq{}}\PY{p}{,} \PY{l+s+s2}{\PYZdq{}}\PY{l+s+s2}{\PYZdq{}}\PY{p}{)}
                                               \PY{o}{.}\PY{n}{replace}\PY{p}{(}\PY{l+s+s2}{\PYZdq{}}\PY{l+s+se}{\PYZbs{}n}\PY{l+s+s2}{\PYZdq{}}\PY{p}{,} \PY{l+s+s2}{\PYZdq{}}\PY{l+s+s2}{\PYZdq{}}\PY{p}{)}\PY{o}{.}\PY{n}{lower}\PY{p}{(}\PY{p}{)}\PY{p}{)}\PY{p}{,} 
                                          \PY{n}{return\PYZus{}counts}\PY{o}{=}\PY{k+kc}{True}\PY{p}{)}
\end{Verbatim}

    \begin{Verbatim}[commandchars=\\\{\}]
{\color{incolor}In [{\color{incolor}10}]:} \PY{n}{xletter\PYZus{}cnt\PYZus{}prob} \PY{o}{=} \PY{n}{xletter\PYZus{}cnt}\PY{o}{/}\PY{n}{xletter\PYZus{}cnt}\PY{o}{.}\PY{n}{sum}\PY{p}{(}\PY{p}{)}
         \PY{n}{yletter\PYZus{}cnt\PYZus{}prob} \PY{o}{=} \PY{n}{yletter\PYZus{}cnt}\PY{o}{/}\PY{n}{yletter\PYZus{}cnt}\PY{o}{.}\PY{n}{sum}\PY{p}{(}\PY{p}{)}
        

\end{Verbatim}

    \begin{center}
    \adjustimage{max size={0.9\linewidth}{0.9\paperheight}}{output_15_0.png}
    \end{center}
    { \hspace*{\fill} \\}
    


    \begin{center}
    \adjustimage{max size={0.9\linewidth}{0.9\paperheight}}{output_16_0.png}
    \end{center}
    { \hspace*{\fill} \\}
    
    \hypertarget{pux159uxedklad-3}{%
\subsubsection{Příklad 3}\label{pux159uxedklad-3}}

\hypertarget{vytvoux159enuxed-kontingenux10dnuxed-tabulky}{%
\paragraph{Vytvoření kontingenční
tabulky}\label{vytvoux159enuxed-kontingenux10dnuxed-tabulky}}

\begin{itemize}
\tightlist
\item
  V případě četností \textless{} 5 se v našem algoritmu podle potřeby
  automaticky spojí biny.
\end{itemize}

\hypertarget{test-samotnuxfd}{%
\paragraph{Test samotný}\label{test-samotnuxfd}}

\begin{itemize}
\tightlist
\item
  Samotný test pak probíhá klasickým způsobem a jeho výstup je znázoněn
  níže.
\end{itemize}




    \begin{Verbatim}[commandchars=\\\{\}]
{\color{incolor}In [{\color{incolor}14}]:} \PY{k}{def} \PY{n+nf}{contingency\PYZus{}table\PYZus{}test}\PY{p}{(}\PY{n}{x\PYZus{}labels}\PY{p}{,} \PY{n}{x\PYZus{}counts}\PY{p}{,} 
                                    \PY{n}{y\PYZus{}labels}\PY{p}{,} \PY{n}{y\PYZus{}counts}\PY{p}{,} 
                                    \PY{n}{a} \PY{o}{=} \PY{l+m+mf}{0.05}\PY{p}{)}\PY{p}{:}
             \PY{c+c1}{\PYZsh{} merge the two series with letters and their counts }
             \PY{c+c1}{\PYZsh{} (convert to virtual contingency table)}
             \PY{c+c1}{\PYZsh{} then merge bins where (Ni. * N.j) / n \PYZlt{} 5}
             \PY{n}{N}\PY{p}{,} \PY{n}{n}\PY{p}{,} \PY{n}{n\PYZus{}letters} \PY{o}{=} \PY{n}{merge\PYZus{}bins}\PY{p}{(}\PY{o}{*}\PY{n}{two\PYZus{}counts\PYZus{}to\PYZus{}one}\PY{p}{(}\PY{n}{x\PYZus{}labels}\PY{p}{,} 
                                                             \PY{n}{x\PYZus{}counts}\PY{p}{,} 
                                                             \PY{n}{y\PYZus{}labels}\PY{p}{,} 
                                                             \PY{n}{y\PYZus{}counts}\PY{p}{)}\PY{p}{)}
             \PY{c+c1}{\PYZsh{} theoretical frequency }
             \PY{n}{npp} \PY{o}{=} \PY{n}{np}\PY{o}{.}\PY{n}{matmul}\PY{p}{(}\PY{n}{np}\PY{o}{.}\PY{n}{sum}\PY{p}{(}\PY{n}{N}\PY{p}{,} \PY{n}{axis} \PY{o}{=} \PY{l+m+mi}{1}\PY{p}{)}\PY{o}{/}\PY{n}{n}\PY{p}{,} \PY{n}{np}\PY{o}{.}\PY{n}{sum}\PY{p}{(}\PY{n}{N}\PY{p}{,} \PY{n}{axis} \PY{o}{=} \PY{l+m+mi}{0}\PY{p}{)}\PY{o}{/}\PY{n}{n}\PY{p}{)}\PY{o}{*}\PY{n}{n}
             \PY{c+c1}{\PYZsh{} count test statictic}
             \PY{n}{chi}\PY{p}{,} \PY{n}{p}\PY{p}{,} \PY{n}{d}\PY{p}{,} \PY{n}{e} \PY{o}{=} \PY{n}{stats}\PY{o}{.}\PY{n}{chi2\PYZus{}contingency}\PY{p}{(}\PY{n}{N}\PY{p}{,} \PY{n}{correction} \PY{o}{=} \PY{k+kc}{False}\PY{p}{)}
             \PY{c+c1}{\PYZsh{} critical value}
             \PY{n}{cval} \PY{o}{=} \PY{n}{stats}\PY{o}{.}\PY{n}{chi2}\PY{o}{.}\PY{n}{isf}\PY{p}{(}\PY{l+m+mf}{0.05}\PY{p}{,} \PY{n}{n\PYZus{}letters}\PY{p}{)}
             \PY{k}{return} \PY{n}{n}\PY{p}{,} \PY{n}{npp}\PY{p}{,} \PY{n}{chi}\PY{p}{,} \PY{n}{cval}\PY{p}{,} \PY{n}{p}\PY{p}{,} \PY{l+s+s2}{\PYZdq{}}\PY{l+s+s2}{Ano}\PY{l+s+s2}{\PYZdq{}} \PY{k}{if} \PY{n}{chi} \PY{o}{\PYZgt{}}\PY{o}{=} \PY{n}{cval} \PY{k}{else} \PY{l+s+s2}{\PYZdq{}}\PY{l+s+s2}{Ne}\PY{l+s+s2}{\PYZdq{}}
\end{Verbatim}


    \begin{Verbatim}[commandchars=\\\{\}]
{\color{incolor}In [{\color{incolor}15}]:} \PY{n}{n}\PY{p}{,} \PY{n}{npp}\PY{p}{,} \PY{n}{chi}\PY{p}{,} \PY{n}{cval}\PY{p}{,} \PY{n}{p}\PY{p}{,} \PY{n}{refuse} \PY{o}{=} \PY{n}{contingency\PYZus{}table\PYZus{}test}\PY{p}{(}\PY{n}{x\PYZus{}length\PYZus{}set}\PY{p}{,} 
                                                               \PY{n}{x\PYZus{}length\PYZus{}counts}\PY{p}{,} 
                                                               \PY{n}{y\PYZus{}length\PYZus{}set}\PY{p}{,} 
                                                               \PY{n}{y\PYZus{}length\PYZus{}counts}\PY{p}{,} 
                                                               \PY{l+m+mf}{0.05}\PY{p}{)}
         \PY{n+nb}{print}\PY{p}{(}\PY{l+s+s2}{\PYZdq{}}\PY{l+s+s2}{Testová statistika:}\PY{l+s+s2}{\PYZdq{}}\PY{p}{,} \PY{n}{chi}\PY{p}{)}
         \PY{n+nb}{print}\PY{p}{(}\PY{l+s+s2}{\PYZdq{}}\PY{l+s+s2}{Kritická hodnota}\PY{l+s+s2}{\PYZdq{}}\PY{p}{,} \PY{n}{cval}\PY{p}{)}
         \PY{n+nb}{print}\PY{p}{(}\PY{l+s+s2}{\PYZdq{}}\PY{l+s+s2}{p\PYZhy{}hodnota:}\PY{l+s+s2}{\PYZdq{}}\PY{p}{,} \PY{n}{p}\PY{p}{)}
         \PY{n+nb}{print}\PY{p}{(}\PY{l+s+s2}{\PYZdq{}}\PY{l+s+s2}{Zamítáme?}\PY{l+s+s2}{\PYZdq{}}\PY{p}{,} \PY{n}{refuse}\PY{p}{)}
\end{Verbatim}

    \begin{Verbatim}[commandchars=\\\{\}]
Testová statistika: 97.14868027055996
Kritická hodnota 23.684791304840576
p-hodnota: 5.9076610786033936e-15
Zamítáme? Ano

    \end{Verbatim}

    Zamítli jsme hypotézu \(H_0\) (tedy nezávislost četností slov
jednotlivých délek) ve prospěch \(H_a\) (tedy rozdělení četností není
nezávislé). \(p\)-hodnota je téměř nulová, tedy test je velmi silný.

Tento test jsme neočekávali s tak jasným výsledkem i z důvodu poměrně
krátkého textu, ale rozdělení četností pro jednotlivé délky slov není
nezávislé. Tedy četnosti délek slov spíš odpovídají konkrétnímu jazyku
než konkrétnímu textu a nebo může jít o text od stejného autora
například.
\newpage
    \hypertarget{pux159uxedklad-4}{%
\subsubsection{Příklad 4}\label{pux159uxedklad-4}}

\hypertarget{testovuxe1nuxed-rovnosti-stux159ednuxedch-hodnot}{%
\paragraph{Testování rovnosti středních
hodnot}\label{testovuxe1nuxed-rovnosti-stux159ednuxedch-hodnot}}

\begin{itemize}
\tightlist
\item
  Jelikož z první úlohy víme, že se rozptyly nerovnají, použili jsme
  test, který nepředpokládá roznost rozptylů.
\item
  Protože je k výpočtu kritické hodnoty potřeba hodnota \(n_d\) jako
  stupně volnosti, nebylo možné jednoduše použít funkci, ale bylo
  potřeba hodnotu manuálně vypočítat.
\end{itemize}

    \begin{Verbatim}[commandchars=\\\{\}]
{\color{incolor}In [{\color{incolor}16}]:} \PY{n}{n} \PY{o}{=} \PY{n+nb}{len}\PY{p}{(}\PY{n}{Xlengths}\PY{p}{)}
         \PY{n}{m} \PY{o}{=} \PY{n+nb}{len}\PY{p}{(}\PY{n}{Ylengths}\PY{p}{)}
         \PY{n}{meanX} \PY{o}{=} \PY{n}{np}\PY{o}{.}\PY{n}{mean}\PY{p}{(}\PY{n}{Xlengths}\PY{p}{)}
         \PY{n}{meanY} \PY{o}{=} \PY{n}{np}\PY{o}{.}\PY{n}{mean}\PY{p}{(}\PY{n}{Ylengths}\PY{p}{)}
         \PY{n}{varX} \PY{o}{=} \PY{n}{np}\PY{o}{.}\PY{n}{var}\PY{p}{(}\PY{n}{Xlengths}\PY{p}{,} \PY{n}{ddof}\PY{o}{=}\PY{l+m+mi}{1}\PY{p}{)}
         \PY{n}{varY} \PY{o}{=} \PY{n}{np}\PY{o}{.}\PY{n}{var}\PY{p}{(}\PY{n}{Ylengths}\PY{p}{,} \PY{n}{ddof}\PY{o}{=}\PY{l+m+mi}{1}\PY{p}{)}
         
         \PY{n}{sd} \PY{o}{=} \PY{n}{np}\PY{o}{.}\PY{n}{sqrt}\PY{p}{(}\PY{n}{varX}\PY{o}{/}\PY{n}{n} \PY{o}{+} \PY{n}{varY}\PY{o}{/}\PY{n}{m}\PY{p}{)}
         \PY{n}{T} \PY{o}{=} \PY{p}{(}\PY{n}{meanX} \PY{o}{\PYZhy{}} \PY{n}{meanY}\PY{p}{)}\PY{o}{/}\PY{n}{sd}
         \PY{n+nb}{print}\PY{p}{(}\PY{l+s+s2}{\PYZdq{}}\PY{l+s+s2}{Testová statistika:}\PY{l+s+s2}{\PYZdq{}}\PY{p}{,} \PY{n}{T}\PY{p}{)}
         
         \PY{n}{nd} \PY{o}{=} \PY{p}{(}\PY{n}{sd}\PY{o}{*}\PY{o}{*}\PY{l+m+mi}{4}\PY{p}{)}\PY{o}{/}\PY{p}{(}\PY{l+m+mi}{1}\PY{o}{/}\PY{p}{(}\PY{n}{n}\PY{o}{\PYZhy{}}\PY{l+m+mi}{1}\PY{p}{)}\PY{o}{*}\PY{p}{(}\PY{n}{varX}\PY{o}{/}\PY{n}{n}\PY{p}{)}\PY{o}{*}\PY{o}{*}\PY{l+m+mi}{2} \PY{o}{+} \PY{l+m+mi}{1}\PY{o}{/}\PY{p}{(}\PY{n}{m}\PY{o}{\PYZhy{}}\PY{l+m+mi}{1}\PY{p}{)}\PY{o}{*}\PY{p}{(}\PY{n}{varY}\PY{o}{/}\PY{n}{m}\PY{p}{)}\PY{o}{*}\PY{o}{*}\PY{l+m+mi}{2}\PY{p}{)}
         
         \PY{n}{p\PYZus{}val} \PY{o}{=} \PY{l+m+mi}{2}\PY{o}{*}\PY{n}{stats}\PY{o}{.}\PY{n}{t}\PY{o}{.}\PY{n}{sf}\PY{p}{(}\PY{n}{np}\PY{o}{.}\PY{n}{abs}\PY{p}{(}\PY{n}{T}\PY{p}{)}\PY{p}{,} \PY{n}{df} \PY{o}{=} \PY{n}{nd}\PY{p}{)}
         \PY{n+nb}{print}\PY{p}{(}\PY{l+s+s2}{\PYZdq{}}\PY{l+s+s2}{p\PYZhy{}hodnota:}\PY{l+s+s2}{\PYZdq{}}\PY{p}{,} \PY{n}{p\PYZus{}val}\PY{p}{)}
         
         
         \PY{n}{krit} \PY{o}{=} \PY{n}{stats}\PY{o}{.}\PY{n}{t}\PY{o}{.}\PY{n}{isf}\PY{p}{(}\PY{l+m+mf}{0.05}\PY{o}{/}\PY{l+m+mi}{2}\PY{p}{,} \PY{n}{df} \PY{o}{=} \PY{n}{nd}\PY{p}{)}
         \PY{n+nb}{print}\PY{p}{(}\PY{l+s+s2}{\PYZdq{}}\PY{l+s+s2}{Kriticka hodnota:}\PY{l+s+s2}{\PYZdq{}}\PY{p}{,} \PY{n}{krit}\PY{p}{)}
         
         \PY{n+nb}{print}\PY{p}{(}\PY{l+s+s2}{\PYZdq{}}\PY{l+s+s2}{Zamítáme:}\PY{l+s+s2}{\PYZdq{}}\PY{p}{,} \PY{l+s+s2}{\PYZdq{}}\PY{l+s+s2}{Ano}\PY{l+s+s2}{\PYZdq{}} \PY{k}{if} \PY{n}{np}\PY{o}{.}\PY{n}{abs}\PY{p}{(}\PY{n}{T}\PY{p}{)} \PY{o}{\PYZgt{}}\PY{o}{=} \PY{n}{krit} \PY{k}{else} \PY{l+s+s2}{\PYZdq{}}\PY{l+s+s2}{Ne}\PY{l+s+s2}{\PYZdq{}}\PY{p}{,} \PY{l+s+s2}{\PYZdq{}}\PY{l+s+se}{\PYZbs{}n}\PY{l+s+s2}{\PYZdq{}}\PY{p}{)}
         
         \PY{c+c1}{\PYZsh{} the easy way but without the critical value}
         \PY{c+c1}{\PYZsh{}print(stats.ttest\PYZus{}ind(Xlengths, Ylengths, equal\PYZus{}var = False))}
\end{Verbatim}

    \begin{Verbatim}[commandchars=\\\{\}]
Testová statistika: -2.784098872865385
p-hodnota: 0.005413212227311425
Kriticka hodnota: 1.9610342501606424
Zamítáme: Ano 


    \end{Verbatim}

    Zamítli jsme hypotézu \(H_0\) (rovnost středních hodnot) ve prospěch
\(H_a\) (tedy, že se střední hodnoty nerovnají). \(p\)-hodnota vyšla 0,5
procenta, což lze také brát za poměrně silný test. Test jsme prováděli s
předpokladen nerovnosti rozptylů.

Z vypočítaných výběrových středních hodnot jsme spíš očekávali rovnost
středních hodnot, avšak test předpoklad vyvrátil, tedy se nerovnají.
\newpage
    \hypertarget{pux159uxedklad-5}{%
\subsubsection{Příklad 5}\label{pux159uxedklad-5}}

\hypertarget{vytvoux159enuxed-kontingenux10dnuxed-tabulky}{%
\paragraph{Vytvoření kontingenční
tabulky}\label{vytvoux159enuxed-kontingenux10dnuxed-tabulky}}

\begin{itemize}
\tightlist
\item
  V případě četností \textless{} 5 se v našem algoritmu podle potřeby
  automaticky spojí biny.
\end{itemize}

\hypertarget{test-samotnuxfd}{%
\paragraph{Test samotný}\label{test-samotnuxfd}}

\begin{itemize}
\tightlist
\item
  Samotný test pak probíhá klasickým způsobem a jeho výstup je znázoněn
  níže.
\end{itemize}

    \begin{Verbatim}[commandchars=\\\{\}]
{\color{incolor}In [{\color{incolor}17}]:} \PY{n}{n}\PY{p}{,} \PY{n}{npp}\PY{p}{,} \PY{n}{chi}\PY{p}{,} \PY{n}{cval}\PY{p}{,} \PY{n}{p}\PY{p}{,} \PY{n}{refuse} \PY{o}{=} \PY{n}{contingency\PYZus{}table\PYZus{}test}\PY{p}{(}\PY{n}{xletters}\PY{p}{,} 
                                                               \PY{n}{xletter\PYZus{}cnt}\PY{p}{,} 
                                                               \PY{n}{yletters}\PY{p}{,} 
                                                               \PY{n}{yletter\PYZus{}cnt}\PY{p}{,} 
                                                               \PY{l+m+mf}{0.05}\PY{p}{)}
         \PY{n+nb}{print}\PY{p}{(}\PY{p}{)}
         \PY{n+nb}{print}\PY{p}{(}\PY{l+s+s2}{\PYZdq{}}\PY{l+s+s2}{Testová statistika:}\PY{l+s+s2}{\PYZdq{}}\PY{p}{,} \PY{n}{chi}\PY{p}{)}
         \PY{n+nb}{print}\PY{p}{(}\PY{l+s+s2}{\PYZdq{}}\PY{l+s+s2}{Kritická hodnota}\PY{l+s+s2}{\PYZdq{}}\PY{p}{,} \PY{n}{cval}\PY{p}{)}
         \PY{n+nb}{print}\PY{p}{(}\PY{l+s+s2}{\PYZdq{}}\PY{l+s+s2}{p\PYZhy{}hodnota:}\PY{l+s+s2}{\PYZdq{}}\PY{p}{,} \PY{n}{p}\PY{p}{)}
         \PY{n+nb}{print}\PY{p}{(}\PY{l+s+s2}{\PYZdq{}}\PY{l+s+s2}{Zamítáme?}\PY{l+s+s2}{\PYZdq{}}\PY{p}{,} \PY{n}{refuse}\PY{p}{)}
\end{Verbatim}

    \begin{Verbatim}[commandchars=\\\{\}]
Merging "." and ","
Merging "a" and ".,"

Testová statistika: 157.1058270988993
Kritická hodnota 38.88513865983007
p-hodnota: 4.0805403234958455e-21
Zamítáme? Ano

    \end{Verbatim}

    Zamítli jsme hypotézu \(H_0\) (tedy nezávislost pravděpodobností písmen
v textech) ve prospěch \(H_a\) (tedy pravděpodobnosti nejsou nezávislá).
\(P\)-hodnota je téměř nulová, tedy test je velmi silný.

Výsledek testu odpovídá očekávání. Očekávali jsme, že rozdělení písmen
bude odpovídat jazyku než jednotlivým textům nebo autorům ve stejném
jazyce.


    % Add a bibliography block to the postdoc
    
    
    
    \end{document}
