
% Default to the notebook output style

    


% Inherit from the specified cell style.




    
\documentclass[11pt]{article}

    
    
    \usepackage[T1]{fontenc}
    % Nicer default font (+ math font) than Computer Modern for most use cases
    \usepackage{mathpazo}

    % Basic figure setup, for now with no caption control since it's done
    % automatically by Pandoc (which extracts ![](path) syntax from Markdown).
    \usepackage{graphicx}
    % We will generate all images so they have a width \maxwidth. This means
    % that they will get their normal width if they fit onto the page, but
    % are scaled down if they would overflow the margins.
    \makeatletter
    \def\maxwidth{\ifdim\Gin@nat@width>\linewidth\linewidth
    \else\Gin@nat@width\fi}
    \makeatother
    \let\Oldincludegraphics\includegraphics
    % Set max figure width to be 80% of text width, for now hardcoded.
    \renewcommand{\includegraphics}[1]{\Oldincludegraphics[width=.8\maxwidth]{#1}}
    % Ensure that by default, figures have no caption (until we provide a
    % proper Figure object with a Caption API and a way to capture that
    % in the conversion process - todo).
    \usepackage{caption}
    \DeclareCaptionLabelFormat{nolabel}{}
    \captionsetup{labelformat=nolabel}
	\usepackage{pdflscape}
    \usepackage{adjustbox} % Used to constrain images to a maximum size 
    \usepackage{xcolor} % Allow colors to be defined
    \usepackage{enumerate} % Needed for markdown enumerations to work
    \usepackage{geometry} % Used to adjust the document margins
    \usepackage{amsmath} % Equations
    \usepackage{amssymb} % Equations
    \usepackage{textcomp} % defines textquotesingle
    % Hack from http://tex.stackexchange.com/a/47451/13684:
    \AtBeginDocument{%
        \def\PYZsq{\textquotesingle}% Upright quotes in Pygmentized code
    }
    \usepackage{upquote} % Upright quotes for verbatim code
    \usepackage{eurosym} % defines \euro
    \usepackage[mathletters]{ucs} % Extended unicode (utf-8) support
    \usepackage[utf8x]{inputenc} % Allow utf-8 characters in the tex document
    \usepackage{fancyvrb} % verbatim replacement that allows latex
    \usepackage{grffile} % extends the file name processing of package graphics 
                         % to support a larger range 
    % The hyperref package gives us a pdf with properly built
    % internal navigation ('pdf bookmarks' for the table of contents,
    % internal cross-reference links, web links for URLs, etc.)
    \usepackage{hyperref}
    \usepackage{longtable} % longtable support required by pandoc >1.10
    \usepackage{booktabs}  % table support for pandoc > 1.12.2
    \usepackage[inline]{enumitem} % IRkernel/repr support (it uses the enumerate* environment)
    \usepackage[normalem]{ulem} % ulem is needed to support strikethroughs (\sout)
                                % normalem makes italics be italics, not underlines
    \usepackage{mathrsfs}
    

    
    
    % Colors for the hyperref package
    \definecolor{urlcolor}{rgb}{0,.145,.698}
    \definecolor{linkcolor}{rgb}{.71,0.21,0.01}
    \definecolor{citecolor}{rgb}{.12,.54,.11}

    % ANSI colors
    \definecolor{ansi-black}{HTML}{3E424D}
    \definecolor{ansi-black-intense}{HTML}{282C36}
    \definecolor{ansi-red}{HTML}{E75C58}
    \definecolor{ansi-red-intense}{HTML}{B22B31}
    \definecolor{ansi-green}{HTML}{00A250}
    \definecolor{ansi-green-intense}{HTML}{007427}
    \definecolor{ansi-yellow}{HTML}{DDB62B}
    \definecolor{ansi-yellow-intense}{HTML}{B27D12}
    \definecolor{ansi-blue}{HTML}{208FFB}
    \definecolor{ansi-blue-intense}{HTML}{0065CA}
    \definecolor{ansi-magenta}{HTML}{D160C4}
    \definecolor{ansi-magenta-intense}{HTML}{A03196}
    \definecolor{ansi-cyan}{HTML}{60C6C8}
    \definecolor{ansi-cyan-intense}{HTML}{258F8F}
    \definecolor{ansi-white}{HTML}{C5C1B4}
    \definecolor{ansi-white-intense}{HTML}{A1A6B2}
    \definecolor{ansi-default-inverse-fg}{HTML}{FFFFFF}
    \definecolor{ansi-default-inverse-bg}{HTML}{000000}

    % commands and environments needed by pandoc snippets
    % extracted from the output of `pandoc -s`
    \providecommand{\tightlist}{%
      \setlength{\itemsep}{0pt}\setlength{\parskip}{0pt}}
    \DefineVerbatimEnvironment{Highlighting}{Verbatim}{commandchars=\\\{\}}
    % Add ',fontsize=\small' for more characters per line
    \newenvironment{Shaded}{}{}
    \newcommand{\KeywordTok}[1]{\textcolor[rgb]{0.00,0.44,0.13}{\textbf{{#1}}}}
    \newcommand{\DataTypeTok}[1]{\textcolor[rgb]{0.56,0.13,0.00}{{#1}}}
    \newcommand{\DecValTok}[1]{\textcolor[rgb]{0.25,0.63,0.44}{{#1}}}
    \newcommand{\BaseNTok}[1]{\textcolor[rgb]{0.25,0.63,0.44}{{#1}}}
    \newcommand{\FloatTok}[1]{\textcolor[rgb]{0.25,0.63,0.44}{{#1}}}
    \newcommand{\CharTok}[1]{\textcolor[rgb]{0.25,0.44,0.63}{{#1}}}
    \newcommand{\StringTok}[1]{\textcolor[rgb]{0.25,0.44,0.63}{{#1}}}
    \newcommand{\CommentTok}[1]{\textcolor[rgb]{0.38,0.63,0.69}{\textit{{#1}}}}
    \newcommand{\OtherTok}[1]{\textcolor[rgb]{0.00,0.44,0.13}{{#1}}}
    \newcommand{\AlertTok}[1]{\textcolor[rgb]{1.00,0.00,0.00}{\textbf{{#1}}}}
    \newcommand{\FunctionTok}[1]{\textcolor[rgb]{0.02,0.16,0.49}{{#1}}}
    \newcommand{\RegionMarkerTok}[1]{{#1}}
    \newcommand{\ErrorTok}[1]{\textcolor[rgb]{1.00,0.00,0.00}{\textbf{{#1}}}}
    \newcommand{\NormalTok}[1]{{#1}}
    
    % Additional commands for more recent versions of Pandoc
    \newcommand{\ConstantTok}[1]{\textcolor[rgb]{0.53,0.00,0.00}{{#1}}}
    \newcommand{\SpecialCharTok}[1]{\textcolor[rgb]{0.25,0.44,0.63}{{#1}}}
    \newcommand{\VerbatimStringTok}[1]{\textcolor[rgb]{0.25,0.44,0.63}{{#1}}}
    \newcommand{\SpecialStringTok}[1]{\textcolor[rgb]{0.73,0.40,0.53}{{#1}}}
    \newcommand{\ImportTok}[1]{{#1}}
    \newcommand{\DocumentationTok}[1]{\textcolor[rgb]{0.73,0.13,0.13}{\textit{{#1}}}}
    \newcommand{\AnnotationTok}[1]{\textcolor[rgb]{0.38,0.63,0.69}{\textbf{\textit{{#1}}}}}
    \newcommand{\CommentVarTok}[1]{\textcolor[rgb]{0.38,0.63,0.69}{\textbf{\textit{{#1}}}}}
    \newcommand{\VariableTok}[1]{\textcolor[rgb]{0.10,0.09,0.49}{{#1}}}
    \newcommand{\ControlFlowTok}[1]{\textcolor[rgb]{0.00,0.44,0.13}{\textbf{{#1}}}}
    \newcommand{\OperatorTok}[1]{\textcolor[rgb]{0.40,0.40,0.40}{{#1}}}
    \newcommand{\BuiltInTok}[1]{{#1}}
    \newcommand{\ExtensionTok}[1]{{#1}}
    \newcommand{\PreprocessorTok}[1]{\textcolor[rgb]{0.74,0.48,0.00}{{#1}}}
    \newcommand{\AttributeTok}[1]{\textcolor[rgb]{0.49,0.56,0.16}{{#1}}}
    \newcommand{\InformationTok}[1]{\textcolor[rgb]{0.38,0.63,0.69}{\textbf{\textit{{#1}}}}}
    \newcommand{\WarningTok}[1]{\textcolor[rgb]{0.38,0.63,0.69}{\textbf{\textit{{#1}}}}}
    
    
    % Define a nice break command that doesn't care if a line doesn't already
    % exist.
    \def\br{\hspace*{\fill} \\* }
    % Math Jax compatibility definitions
    \def\gt{>}
    \def\lt{<}
    \let\Oldtex\TeX
    \let\Oldlatex\LaTeX
    \renewcommand{\TeX}{\textrm{\Oldtex}}
    \renewcommand{\LaTeX}{\textrm{\Oldlatex}}
    % Document parameters
    % Document title
    \title{Domácí úkol 2}
    \author{Ladislav Martínek a Richard Werner}
    
    
    
    
    

    % Pygments definitions
    
\makeatletter
\def\PY@reset{\let\PY@it=\relax \let\PY@bf=\relax%
    \let\PY@ul=\relax \let\PY@tc=\relax%
    \let\PY@bc=\relax \let\PY@ff=\relax}
\def\PY@tok#1{\csname PY@tok@#1\endcsname}
\def\PY@toks#1+{\ifx\relax#1\empty\else%
    \PY@tok{#1}\expandafter\PY@toks\fi}
\def\PY@do#1{\PY@bc{\PY@tc{\PY@ul{%
    \PY@it{\PY@bf{\PY@ff{#1}}}}}}}
\def\PY#1#2{\PY@reset\PY@toks#1+\relax+\PY@do{#2}}

\expandafter\def\csname PY@tok@w\endcsname{\def\PY@tc##1{\textcolor[rgb]{0.73,0.73,0.73}{##1}}}
\expandafter\def\csname PY@tok@c\endcsname{\let\PY@it=\textit\def\PY@tc##1{\textcolor[rgb]{0.25,0.50,0.50}{##1}}}
\expandafter\def\csname PY@tok@cp\endcsname{\def\PY@tc##1{\textcolor[rgb]{0.74,0.48,0.00}{##1}}}
\expandafter\def\csname PY@tok@k\endcsname{\let\PY@bf=\textbf\def\PY@tc##1{\textcolor[rgb]{0.00,0.50,0.00}{##1}}}
\expandafter\def\csname PY@tok@kp\endcsname{\def\PY@tc##1{\textcolor[rgb]{0.00,0.50,0.00}{##1}}}
\expandafter\def\csname PY@tok@kt\endcsname{\def\PY@tc##1{\textcolor[rgb]{0.69,0.00,0.25}{##1}}}
\expandafter\def\csname PY@tok@o\endcsname{\def\PY@tc##1{\textcolor[rgb]{0.40,0.40,0.40}{##1}}}
\expandafter\def\csname PY@tok@ow\endcsname{\let\PY@bf=\textbf\def\PY@tc##1{\textcolor[rgb]{0.67,0.13,1.00}{##1}}}
\expandafter\def\csname PY@tok@nb\endcsname{\def\PY@tc##1{\textcolor[rgb]{0.00,0.50,0.00}{##1}}}
\expandafter\def\csname PY@tok@nf\endcsname{\def\PY@tc##1{\textcolor[rgb]{0.00,0.00,1.00}{##1}}}
\expandafter\def\csname PY@tok@nc\endcsname{\let\PY@bf=\textbf\def\PY@tc##1{\textcolor[rgb]{0.00,0.00,1.00}{##1}}}
\expandafter\def\csname PY@tok@nn\endcsname{\let\PY@bf=\textbf\def\PY@tc##1{\textcolor[rgb]{0.00,0.00,1.00}{##1}}}
\expandafter\def\csname PY@tok@ne\endcsname{\let\PY@bf=\textbf\def\PY@tc##1{\textcolor[rgb]{0.82,0.25,0.23}{##1}}}
\expandafter\def\csname PY@tok@nv\endcsname{\def\PY@tc##1{\textcolor[rgb]{0.10,0.09,0.49}{##1}}}
\expandafter\def\csname PY@tok@no\endcsname{\def\PY@tc##1{\textcolor[rgb]{0.53,0.00,0.00}{##1}}}
\expandafter\def\csname PY@tok@nl\endcsname{\def\PY@tc##1{\textcolor[rgb]{0.63,0.63,0.00}{##1}}}
\expandafter\def\csname PY@tok@ni\endcsname{\let\PY@bf=\textbf\def\PY@tc##1{\textcolor[rgb]{0.60,0.60,0.60}{##1}}}
\expandafter\def\csname PY@tok@na\endcsname{\def\PY@tc##1{\textcolor[rgb]{0.49,0.56,0.16}{##1}}}
\expandafter\def\csname PY@tok@nt\endcsname{\let\PY@bf=\textbf\def\PY@tc##1{\textcolor[rgb]{0.00,0.50,0.00}{##1}}}
\expandafter\def\csname PY@tok@nd\endcsname{\def\PY@tc##1{\textcolor[rgb]{0.67,0.13,1.00}{##1}}}
\expandafter\def\csname PY@tok@s\endcsname{\def\PY@tc##1{\textcolor[rgb]{0.73,0.13,0.13}{##1}}}
\expandafter\def\csname PY@tok@sd\endcsname{\let\PY@it=\textit\def\PY@tc##1{\textcolor[rgb]{0.73,0.13,0.13}{##1}}}
\expandafter\def\csname PY@tok@si\endcsname{\let\PY@bf=\textbf\def\PY@tc##1{\textcolor[rgb]{0.73,0.40,0.53}{##1}}}
\expandafter\def\csname PY@tok@se\endcsname{\let\PY@bf=\textbf\def\PY@tc##1{\textcolor[rgb]{0.73,0.40,0.13}{##1}}}
\expandafter\def\csname PY@tok@sr\endcsname{\def\PY@tc##1{\textcolor[rgb]{0.73,0.40,0.53}{##1}}}
\expandafter\def\csname PY@tok@ss\endcsname{\def\PY@tc##1{\textcolor[rgb]{0.10,0.09,0.49}{##1}}}
\expandafter\def\csname PY@tok@sx\endcsname{\def\PY@tc##1{\textcolor[rgb]{0.00,0.50,0.00}{##1}}}
\expandafter\def\csname PY@tok@m\endcsname{\def\PY@tc##1{\textcolor[rgb]{0.40,0.40,0.40}{##1}}}
\expandafter\def\csname PY@tok@gh\endcsname{\let\PY@bf=\textbf\def\PY@tc##1{\textcolor[rgb]{0.00,0.00,0.50}{##1}}}
\expandafter\def\csname PY@tok@gu\endcsname{\let\PY@bf=\textbf\def\PY@tc##1{\textcolor[rgb]{0.50,0.00,0.50}{##1}}}
\expandafter\def\csname PY@tok@gd\endcsname{\def\PY@tc##1{\textcolor[rgb]{0.63,0.00,0.00}{##1}}}
\expandafter\def\csname PY@tok@gi\endcsname{\def\PY@tc##1{\textcolor[rgb]{0.00,0.63,0.00}{##1}}}
\expandafter\def\csname PY@tok@gr\endcsname{\def\PY@tc##1{\textcolor[rgb]{1.00,0.00,0.00}{##1}}}
\expandafter\def\csname PY@tok@ge\endcsname{\let\PY@it=\textit}
\expandafter\def\csname PY@tok@gs\endcsname{\let\PY@bf=\textbf}
\expandafter\def\csname PY@tok@gp\endcsname{\let\PY@bf=\textbf\def\PY@tc##1{\textcolor[rgb]{0.00,0.00,0.50}{##1}}}
\expandafter\def\csname PY@tok@go\endcsname{\def\PY@tc##1{\textcolor[rgb]{0.53,0.53,0.53}{##1}}}
\expandafter\def\csname PY@tok@gt\endcsname{\def\PY@tc##1{\textcolor[rgb]{0.00,0.27,0.87}{##1}}}
\expandafter\def\csname PY@tok@err\endcsname{\def\PY@bc##1{\setlength{\fboxsep}{0pt}\fcolorbox[rgb]{1.00,0.00,0.00}{1,1,1}{\strut ##1}}}
\expandafter\def\csname PY@tok@kc\endcsname{\let\PY@bf=\textbf\def\PY@tc##1{\textcolor[rgb]{0.00,0.50,0.00}{##1}}}
\expandafter\def\csname PY@tok@kd\endcsname{\let\PY@bf=\textbf\def\PY@tc##1{\textcolor[rgb]{0.00,0.50,0.00}{##1}}}
\expandafter\def\csname PY@tok@kn\endcsname{\let\PY@bf=\textbf\def\PY@tc##1{\textcolor[rgb]{0.00,0.50,0.00}{##1}}}
\expandafter\def\csname PY@tok@kr\endcsname{\let\PY@bf=\textbf\def\PY@tc##1{\textcolor[rgb]{0.00,0.50,0.00}{##1}}}
\expandafter\def\csname PY@tok@bp\endcsname{\def\PY@tc##1{\textcolor[rgb]{0.00,0.50,0.00}{##1}}}
\expandafter\def\csname PY@tok@fm\endcsname{\def\PY@tc##1{\textcolor[rgb]{0.00,0.00,1.00}{##1}}}
\expandafter\def\csname PY@tok@vc\endcsname{\def\PY@tc##1{\textcolor[rgb]{0.10,0.09,0.49}{##1}}}
\expandafter\def\csname PY@tok@vg\endcsname{\def\PY@tc##1{\textcolor[rgb]{0.10,0.09,0.49}{##1}}}
\expandafter\def\csname PY@tok@vi\endcsname{\def\PY@tc##1{\textcolor[rgb]{0.10,0.09,0.49}{##1}}}
\expandafter\def\csname PY@tok@vm\endcsname{\def\PY@tc##1{\textcolor[rgb]{0.10,0.09,0.49}{##1}}}
\expandafter\def\csname PY@tok@sa\endcsname{\def\PY@tc##1{\textcolor[rgb]{0.73,0.13,0.13}{##1}}}
\expandafter\def\csname PY@tok@sb\endcsname{\def\PY@tc##1{\textcolor[rgb]{0.73,0.13,0.13}{##1}}}
\expandafter\def\csname PY@tok@sc\endcsname{\def\PY@tc##1{\textcolor[rgb]{0.73,0.13,0.13}{##1}}}
\expandafter\def\csname PY@tok@dl\endcsname{\def\PY@tc##1{\textcolor[rgb]{0.73,0.13,0.13}{##1}}}
\expandafter\def\csname PY@tok@s2\endcsname{\def\PY@tc##1{\textcolor[rgb]{0.73,0.13,0.13}{##1}}}
\expandafter\def\csname PY@tok@sh\endcsname{\def\PY@tc##1{\textcolor[rgb]{0.73,0.13,0.13}{##1}}}
\expandafter\def\csname PY@tok@s1\endcsname{\def\PY@tc##1{\textcolor[rgb]{0.73,0.13,0.13}{##1}}}
\expandafter\def\csname PY@tok@mb\endcsname{\def\PY@tc##1{\textcolor[rgb]{0.40,0.40,0.40}{##1}}}
\expandafter\def\csname PY@tok@mf\endcsname{\def\PY@tc##1{\textcolor[rgb]{0.40,0.40,0.40}{##1}}}
\expandafter\def\csname PY@tok@mh\endcsname{\def\PY@tc##1{\textcolor[rgb]{0.40,0.40,0.40}{##1}}}
\expandafter\def\csname PY@tok@mi\endcsname{\def\PY@tc##1{\textcolor[rgb]{0.40,0.40,0.40}{##1}}}
\expandafter\def\csname PY@tok@il\endcsname{\def\PY@tc##1{\textcolor[rgb]{0.40,0.40,0.40}{##1}}}
\expandafter\def\csname PY@tok@mo\endcsname{\def\PY@tc##1{\textcolor[rgb]{0.40,0.40,0.40}{##1}}}
\expandafter\def\csname PY@tok@ch\endcsname{\let\PY@it=\textit\def\PY@tc##1{\textcolor[rgb]{0.25,0.50,0.50}{##1}}}
\expandafter\def\csname PY@tok@cm\endcsname{\let\PY@it=\textit\def\PY@tc##1{\textcolor[rgb]{0.25,0.50,0.50}{##1}}}
\expandafter\def\csname PY@tok@cpf\endcsname{\let\PY@it=\textit\def\PY@tc##1{\textcolor[rgb]{0.25,0.50,0.50}{##1}}}
\expandafter\def\csname PY@tok@c1\endcsname{\let\PY@it=\textit\def\PY@tc##1{\textcolor[rgb]{0.25,0.50,0.50}{##1}}}
\expandafter\def\csname PY@tok@cs\endcsname{\let\PY@it=\textit\def\PY@tc##1{\textcolor[rgb]{0.25,0.50,0.50}{##1}}}

\def\PYZbs{\char`\\}
\def\PYZus{\char`\_}
\def\PYZob{\char`\{}
\def\PYZcb{\char`\}}
\def\PYZca{\char`\^}
\def\PYZam{\char`\&}
\def\PYZlt{\char`\<}
\def\PYZgt{\char`\>}
\def\PYZsh{\char`\#}
\def\PYZpc{\char`\%}
\def\PYZdl{\char`\$}
\def\PYZhy{\char`\-}
\def\PYZsq{\char`\'}
\def\PYZdq{\char`\"}
\def\PYZti{\char`\~}
% for compatibility with earlier versions
\def\PYZat{@}
\def\PYZlb{[}
\def\PYZrb{]}
\makeatother


    % Exact colors from NB
    \definecolor{incolor}{rgb}{0.0, 0.0, 0.5}
    \definecolor{outcolor}{rgb}{0.545, 0.0, 0.0}



    
    % Prevent overflowing lines due to hard-to-break entities
    \sloppy 
    % Setup hyperref package
    \hypersetup{
      breaklinks=true,  % so long urls are correctly broken across lines
      colorlinks=true,
      urlcolor=urlcolor,
      linkcolor=linkcolor,
      citecolor=citecolor,
      }
    % Slightly bigger margins than the latex defaults
    
    \geometry{verbose,tmargin=1in,bmargin=1in,lmargin=1in,rmargin=1in}
    
    

    \begin{document}
    
    
    \maketitle
    
    

    
    \section*{Domácí úkol 2 (6
bodů)}\label{domuxe1cuxed-uxfakol-1-6-bodux16f}

    \section{Úkoly}\label{uxfakoly}

\begin{enumerate}
\def\labelenumi{\arabic{enumi}.}
\tightlist
\item
  (1b) Z datového souboru načtěte text k analýze. Odhadněte
  pravděpodobnosti písmen (včetně mezer), které se v textu vyskytují.
  Takto získané empirické rozdělení graficky znázorněte. Pro další body
  předpokládejme, že je text vygenerován z homogenního markovského
  řetězce s diskrétním časem.
\item
  (1.5b) Za tohoto předpokladu odhadněte matici přechodu.
\item
  (2b) Na základě matice z předchozího bodu najděte stacionární
  rozdělení tohoto řetězce.
\item
  (1.5b) Porovnejte stacionární rozdělení se získaným empirickým
  rozdělením. Tj. na hladině 5\% otestujte hypotézu, že se empirické
  rozdělení z bodu 1 rovná stacionárnímu rozdělení.
\end{enumerate}
\pagebreak
    \section{Řešení}\label{ux159eux161enuxed}

    \begin{Verbatim}[commandchars=\\\{\}]
{\color{incolor}In [{\color{incolor}1}]:} \PY{k+kn}{import} \PY{n+nn}{math}
        \PY{k+kn}{import} \PY{n+nn}{scipy}
        \PY{k+kn}{import} \PY{n+nn}{numpy} \PY{k}{as} \PY{n+nn}{np}
        \PY{k+kn}{import} \PY{n+nn}{matplotlib}\PY{n+nn}{.}\PY{n+nn}{pyplot} \PY{k}{as} \PY{n+nn}{plt}
        \PY{k+kn}{from} \PY{n+nn}{scipy} \PY{k}{import} \PY{n}{stats}
\end{Verbatim}

    \subsection{Zpracování souborů}\label{zpracovuxe1nuxed-souborux16f}

    \begin{Verbatim}[commandchars=\\\{\}]
{\color{incolor}In [{\color{incolor}2}]:} \PY{n}{K} \PY{o}{=} \PY{l+m+mi}{15}
        \PY{n}{L} \PY{o}{=} \PY{n+nb}{len}\PY{p}{(}\PY{l+s+s2}{\PYZdq{}}\PY{l+s+s2}{Martínek}\PY{l+s+s2}{\PYZdq{}}\PY{p}{)}
        \PY{n}{X} \PY{o}{=} \PY{p}{(}\PY{p}{(}\PY{n}{K}\PY{o}{*}\PY{n}{L}\PY{o}{*}\PY{l+m+mi}{23}\PY{p}{)} \PY{o}{\PYZpc{}} \PY{p}{(}\PY{l+m+mi}{20}\PY{p}{)}\PY{p}{)} \PY{o}{+} \PY{l+m+mi}{1}
        \PY{n}{X\PYZus{}file} \PY{o}{=} \PY{l+s+s1}{\PYZsq{}}\PY{l+s+s1}{0}\PY{l+s+s1}{\PYZsq{}}\PY{o}{*}\PY{p}{(}\PY{l+m+mi}{3}\PY{o}{\PYZhy{}}\PY{n+nb}{len}\PY{p}{(}\PY{n+nb}{str}\PY{p}{(}\PY{n}{X}\PY{p}{)}\PY{p}{)}\PY{p}{)}\PY{o}{+}\PY{n+nb}{str}\PY{p}{(}\PY{n}{X}\PY{p}{)}\PY{o}{+}\PY{l+s+s1}{\PYZsq{}}\PY{l+s+s1}{.txt}\PY{l+s+s1}{\PYZsq{}}
\end{Verbatim}

    \begin{Verbatim}[commandchars=\\\{\}]
{\color{incolor}In [{\color{incolor}3}]:} \PY{k}{def} \PY{n+nf}{read\PYZus{}whole\PYZus{}file}\PY{p}{(}\PY{n}{filename}\PY{p}{)}\PY{p}{:}
            \PY{k}{with} \PY{n+nb}{open}\PY{p}{(}\PY{n}{filename}\PY{p}{,} \PY{l+s+s1}{\PYZsq{}}\PY{l+s+s1}{r}\PY{l+s+s1}{\PYZsq{}}\PY{p}{)} \PY{k}{as} \PY{n}{file}\PY{p}{:}
                \PY{k}{return} \PY{n}{file}\PY{o}{.}\PY{n}{read}\PY{p}{(}\PY{p}{)}
        
        \PY{n}{xfile} \PY{o}{=} \PY{n}{read\PYZus{}whole\PYZus{}file}\PY{p}{(}\PY{n}{X\PYZus{}file}\PY{p}{)}
\end{Verbatim}

    \subsection{Příklad 1}\label{pux159uxedklad-1}

    \begin{itemize}
\tightlist
\item
  Na následujícím grafu jsou znázorněny vypočítané pravděpodobnosti
  výskytu daných písmen (znaků) v datovém souboru.
\end{itemize}

    \begin{Verbatim}[commandchars=\\\{\}]
{\color{incolor}In [{\color{incolor}4}]:} \PY{n}{xletters}\PY{p}{,} \PY{n}{xletter\PYZus{}cnt} \PY{o}{=} \PY{n}{np}\PY{o}{.}\PY{n}{unique}\PY{p}{(}\PY{n+nb}{list}\PY{p}{(}\PY{n}{xfile}\PY{o}{.}\PY{n}{replace}\PY{p}{(}\PY{l+s+s2}{\PYZdq{}}\PY{l+s+se}{\PYZbs{}n}\PY{l+s+s2}{\PYZdq{}}\PY{p}{,} \PY{l+s+s2}{\PYZdq{}}\PY{l+s+s2}{\PYZdq{}}\PY{p}{)}\PY{o}{.}\PY{n}{lower}\PY{p}{(}\PY{p}{)}\PY{p}{)}\PY{p}{,} 
						\PY{n}{return\PYZus{}counts}\PY{o}{=}\PY{k+kc}{True}\PY{p}{)}                
\end{Verbatim}

    \begin{Verbatim}[commandchars=\\\{\}]
{\color{incolor}In [{\color{incolor}5}]:} \PY{n}{xletter\PYZus{}cnt\PYZus{}prob} \PY{o}{=} \PY{n}{xletter\PYZus{}cnt}\PY{o}{/}\PY{n}{xletter\PYZus{}cnt}\PY{o}{.}\PY{n}{sum}\PY{p}{(}\PY{p}{)}
        \PY{n}{f}\PY{p}{,} \PY{n}{ax1} \PY{o}{=} \PY{n}{plt}\PY{o}{.}\PY{n}{subplots}\PY{p}{(}\PY{l+m+mi}{1}\PY{p}{,} \PY{l+m+mi}{1}\PY{p}{,} \PY{n}{figsize}\PY{o}{=}\PY{p}{(}\PY{l+m+mi}{15}\PY{p}{,}\PY{l+m+mi}{5}\PY{p}{)}\PY{p}{)}
        \PY{n}{ax1}\PY{o}{.}\PY{n}{bar}\PY{p}{(}\PY{n}{xletters}\PY{p}{,} \PY{n}{xletter\PYZus{}cnt\PYZus{}prob}\PY{p}{,} \PY{n}{color}\PY{o}{=}\PY{p}{(}\PY{l+m+mi}{0}\PY{p}{,} \PY{l+m+mi}{1}\PY{p}{,} \PY{l+m+mf}{0.48}\PY{p}{,} \PY{l+m+mi}{1}\PY{p}{)}\PY{p}{)}
        \PY{k}{for} \PY{n}{tick} \PY{o+ow}{in} \PY{n}{ax1}\PY{o}{.}\PY{n}{xaxis}\PY{o}{.}\PY{n}{get\PYZus{}major\PYZus{}ticks}\PY{p}{(}\PY{p}{)}\PY{p}{:} \PY{n}{tick}\PY{o}{.}\PY{n}{label}\PY{o}{.}\PY{n}{set\PYZus{}fontsize}\PY{p}{(}\PY{l+m+mi}{14}\PY{p}{)} 
        \PY{k}{for} \PY{n}{tick} \PY{o+ow}{in} \PY{n}{ax1}\PY{o}{.}\PY{n}{yaxis}\PY{o}{.}\PY{n}{get\PYZus{}major\PYZus{}ticks}\PY{p}{(}\PY{p}{)}\PY{p}{:} \PY{n}{tick}\PY{o}{.}\PY{n}{label}\PY{o}{.}\PY{n}{set\PYZus{}fontsize}\PY{p}{(}\PY{l+m+mi}{14}\PY{p}{)}
        \PY{n}{ax1}\PY{o}{.}\PY{n}{set\PYZus{}xlabel}\PY{p}{(}\PY{l+s+s1}{\PYZsq{}}\PY{l+s+s1}{znak}\PY{l+s+s1}{\PYZsq{}}\PY{p}{,} \PY{n}{fontsize}\PY{o}{=}\PY{l+m+mi}{16}\PY{p}{)}
        \PY{n}{ax1}\PY{o}{.}\PY{n}{set\PYZus{}ylabel}\PY{p}{(}\PY{l+s+s1}{\PYZsq{}}\PY{l+s+s1}{pravděpodobnost}\PY{l+s+s1}{\PYZsq{}}\PY{p}{,} \PY{n}{fontsize}\PY{o}{=}\PY{l+m+mi}{16}\PY{p}{)}
        \PY{n}{ax1}\PY{o}{.}\PY{n}{set\PYZus{}title}\PY{p}{(}\PY{l+s+s1}{\PYZsq{}}\PY{l+s+s1}{Soubor X}\PY{l+s+s1}{\PYZsq{}}\PY{p}{,} \PY{n}{fontsize}\PY{o}{=}\PY{l+m+mi}{22}\PY{p}{)}
        \PY{n}{plt}\PY{o}{.}\PY{n}{show}\PY{p}{(}\PY{p}{)}
\end{Verbatim}

    \begin{center}
    \adjustimage{max size={0.9\linewidth}{0.9\paperheight}}{output_10_0.png}
    \end{center}
    { \hspace*{\fill} \\}
    
    \subsection{Příklad 2}\label{pux159uxedklad-2}

    Iterovali jsme přes celý textový soubor a napočítali konkrétní přechody
mezi písmeny (znaky). Abychom získali matici přechodů, bylo nutné, aby
suma v řádku byla rovna 1 pro pravděpodobnost přechodu ze znaku $i$ do
znaku $j$. Takže jsme každý řádek matice vydělili jeho součtem. Níže je
vidět výsledná matice přechodu.

    \begin{Verbatim}[commandchars=\\\{\}]
{\color{incolor}In [{\color{incolor}22}]:} \PY{c+c1}{\PYZsh{} dict with keys to the matrix}
         \PY{n}{letters} \PY{o}{=} \PY{n+nb}{list}\PY{p}{(}\PY{n}{xfile}\PY{o}{.}\PY{n}{replace}\PY{p}{(}\PY{l+s+s2}{\PYZdq{}}\PY{l+s+se}{\PYZbs{}n}\PY{l+s+s2}{\PYZdq{}}\PY{p}{,} \PY{l+s+s2}{\PYZdq{}}\PY{l+s+s2}{\PYZdq{}}\PY{p}{)}\PY{o}{.}\PY{n}{lower}\PY{p}{(}\PY{p}{)}\PY{p}{)}
         \PY{n}{letter\PYZus{}to\PYZus{}key} \PY{o}{=} \PY{p}{\PYZob{}}\PY{p}{\PYZcb{}}
         \PY{n}{key\PYZus{}to\PYZus{}letter} \PY{o}{=} \PY{p}{\PYZob{}}\PY{p}{\PYZcb{}}
         \PY{n}{cnt} \PY{o}{=} \PY{l+m+mi}{0}
         \PY{k}{for} \PY{n}{i} \PY{o+ow}{in} \PY{n}{letters}\PY{p}{:}
             \PY{k}{if} \PY{n}{i} \PY{o+ow}{not} \PY{o+ow}{in} \PY{n}{letter\PYZus{}to\PYZus{}key}\PY{p}{:}
                 \PY{n}{letter\PYZus{}to\PYZus{}key}\PY{p}{[}\PY{n}{i}\PY{p}{]} \PY{o}{=} \PY{n}{cnt}
                 \PY{n}{key\PYZus{}to\PYZus{}letter}\PY{p}{[}\PY{n}{cnt}\PY{p}{]} \PY{o}{=} \PY{n}{i}
                 \PY{n}{cnt} \PY{o}{+}\PY{o}{=} \PY{l+m+mi}{1}
         \PY{n}{matrix} \PY{o}{=} \PY{n}{np}\PY{o}{.}\PY{n}{zeros}\PY{p}{(}\PY{p}{(}\PY{n+nb}{len}\PY{p}{(}\PY{n}{letter\PYZus{}to\PYZus{}key}\PY{p}{)}\PY{p}{,} \PY{n+nb}{len}\PY{p}{(}\PY{n}{letter\PYZus{}to\PYZus{}key}\PY{p}{)}\PY{p}{)}\PY{p}{)}
         \PY{c+c1}{\PYZsh{} counts in matrix}
         \PY{k}{for} \PY{n}{i}\PY{p}{,} \PY{n}{letter} \PY{o+ow}{in} \PY{n+nb}{enumerate}\PY{p}{(}\PY{n}{letters}\PY{p}{[}\PY{p}{:}\PY{o}{\PYZhy{}}\PY{l+m+mi}{1}\PY{p}{]}\PY{p}{)}\PY{p}{:}
             \PY{n}{matrix}\PY{p}{[}\PY{n}{letter\PYZus{}to\PYZus{}key}\PY{p}{[}\PY{n}{letter}\PY{p}{]}\PY{p}{]}\PY{p}{[}\PY{n}{letter\PYZus{}to\PYZus{}key}\PY{p}{[}\PY{n}{letters}\PY{p}{[}\PY{n}{i}\PY{o}{+}\PY{l+m+mi}{1}\PY{p}{]}\PY{p}{]}\PY{p}{]} \PY{o}{+}\PY{o}{=} \PY{l+m+mi}{1}
             
         \PY{c+c1}{\PYZsh{} transition matrix \PYZhy{} need sum of row == 1}
         \PY{n}{P} \PY{o}{=} \PY{n}{matrix}\PY{o}{/}\PY{n}{np}\PY{o}{.}\PY{n}{sum}\PY{p}{(}\PY{n}{matrix}\PY{p}{,} \PY{l+m+mi}{1}\PY{p}{)}\PY{p}{[}\PY{p}{:}\PY{p}{,} \PY{n}{np}\PY{o}{.}\PY{n}{newaxis}\PY{p}{]}
         
         \PY{n}{string} \PY{o}{=} \PY{l+s+s1}{\PYZsq{}}\PY{l+s+s1}{ |}\PY{l+s+s1}{\PYZsq{}}
         \PY{k}{for} \PY{n}{j} \PY{o+ow}{in} \PY{n}{key\PYZus{}to\PYZus{}letter}\PY{p}{:}
             \PY{n}{string} \PY{o}{=} \PY{n}{string} \PY{o}{+} \PY{l+s+s1}{\PYZsq{}}\PY{l+s+s1}{ }\PY{l+s+si}{\PYZob{}\PYZcb{}}\PY{l+s+s1}{ |}\PY{l+s+s1}{\PYZsq{}}\PY{o}{.}\PY{n}{format}\PY{p}{(}\PY{n}{key\PYZus{}to\PYZus{}letter}\PY{p}{[}\PY{n}{j}\PY{p}{]}\PY{p}{)}
         \PY{n}{display}\PY{p}{(}\PY{n}{string}\PY{p}{)}
             
         \PY{c+c1}{\PYZsh{} print the matrix}
         \PY{k}{for} \PY{n}{i}\PY{p}{,} \PY{n}{row} \PY{o+ow}{in} \PY{n+nb}{enumerate}\PY{p}{(}\PY{n}{P}\PY{p}{)}\PY{p}{:}
             \PY{n}{string} \PY{o}{=} \PY{n}{key\PYZus{}to\PYZus{}letter}\PY{p}{[}\PY{n}{i}\PY{p}{]}\PY{o}{+}\PY{l+s+s1}{\PYZsq{}}\PY{l+s+s1}{| }\PY{l+s+s1}{\PYZsq{}}
             \PY{k}{for} \PY{n}{j}\PY{p}{,} \PY{n}{letter\PYZus{}cnt} \PY{o+ow}{in} \PY{n+nb}{enumerate}\PY{p}{(}\PY{n}{row}\PY{p}{)}\PY{p}{:}
                 \PY{k}{if} \PY{n+nb}{round}\PY{p}{(}\PY{n}{letter\PYZus{}cnt}\PY{p}{,} \PY{l+m+mi}{2}\PY{p}{)} \PY{o}{==} \PY{l+m+mi}{0}\PY{p}{:}
                     \PY{n}{string} \PY{o}{=} \PY{n}{string} \PY{o}{+} \PY{l+s+s1}{\PYZsq{}}\PY{l+s+si}{\PYZob{}0:.2f\PYZcb{}}\PY{l+s+s1}{|}\PY{l+s+s1}{\PYZsq{}}\PY{o}{.}\PY{n}{format}\PY{p}{(}\PY{n+nb}{round}\PY{p}{(}\PY{n}{letter\PYZus{}cnt}\PY{p}{,} \PY{l+m+mi}{2}\PY{p}{)}\PY{p}{)}
                 \PY{k}{else}\PY{p}{:}
                     \PY{n}{string} \PY{o}{=} \PY{n}{string} \PY{o}{+} \PY{l+s+s1}{\PYZsq{}}\PY{l+s+si}{\PYZob{}0:.2f\PYZcb{}}\PY{l+s+s1}{|}\PY{l+s+s1}{\PYZsq{}}\PY{o}{.}\PY{n}{format}\PY{p}{(}\PY{n+nb}{round}\PY{p}{(}\PY{n}{letter\PYZus{}cnt}\PY{p}{,} \PY{l+m+mi}{2}\PY{p}{)}\PY{p}{)}
             \PY{n}{display}\PY{p}{(}\PY{n}{string}\PY{p}{)}
\end{Verbatim}

\pagebreak
    
    
   \pagestyle{empty}
\begin{landscape}
\begin{table}[]
\begin{center}
% Please add the following required packages to your document preamble:
% \usepackage[table,xcdraw]{xcolor}
% If you use beamer only pass "xcolor=table" option, i.e. \documentclass[xcolor=table]{beamer}
\resizebox{\columnwidth}{200pt}{%
\begin{tabular}{|l||l|l|l|l|l|l|l|l|l|l|l|l|l|l|l|l|l|l|l|l|l|l|l|l|l|l|l|l|l|}
\hline
 & p    & i    & e    & r    & o    & t    & ,    &      & d    & g    & f    & b    & l    & u    & m    & y    & w    & a    & .    & h    & c    & n    & s    & v    & k    & q    & j    & x    & z      \\ \hline \hline
p                         & 0.11 & 0.13 & 0.25 & 0.07 & 0.05 & 0.02 & 0.00 & 0.08 & 0.00 & 0.00 & 0.00 & 0.00 & 0.15 & 0.07 & 0.00 & 0.02 & 0.00 & 0.05 & 0.00 & 0.00 & 0.00 & 0.00 & 0.00 & 0.00 & 0.00 & 0.00 & 0.00 & 0.00 & 0.00   \\ \hline
i                        & 0.01 & 0.00 & 0.08 & 0.08 & 0.02 & 0.15 & 0.00 & 0.06 & 0.03 & 0.04 & 0.03 & 0.00 & 0.03 & 0.01 & 0.05 & 0.00 & 0.00 & 0.01 & 0.00 & 0.00 & 0.04 & 0.24 & 0.09 & 0.02 & 0.01 & 0.00 & 0.00 & 0.00 & 0.00   \\ \hline
e                        & 0.01 & 0.02 & 0.03 & 0.15 & 0.00 & 0.02 & 0.00 & 0.35 & 0.10 & 0.01 & 0.00 & 0.00 & 0.04 & 0.00 & 0.01 & 0.02 & 0.00 & 0.07 & 0.00 & 0.00 & 0.01 & 0.09 & 0.05 & 0.01 & 0.00 & 0.00 & 0.00 & 0.00 & 0.00   \\ \hline
r                        & 0.00 & 0.12 & 0.25 & 0.04 & 0.11 & 0.04 & 0.00 & 0.18 & 0.03 & 0.01 & 0.01 & 0.00 & 0.01 & 0.01 & 0.01 & 0.02 & 0.00 & 0.06 & 0.00 & 0.00 & 0.00 & 0.03 & 0.04 & 0.00 & 0.02 & 0.00 & 0.00 & 0.00 & 0.00   \\ \hline
o                        & 0.01 & 0.01 & 0.00 & 0.11 & 0.05 & 0.07 & 0.00 & 0.12 & 0.01 & 0.05 & 0.11 & 0.00 & 0.05 & 0.14 & 0.06 & 0.00 & 0.03 & 0.00 & 0.00 & 0.00 & 0.00 & 0.11 & 0.02 & 0.01 & 0.01 & 0.00 & 0.00 & 0.00 & 0.01   \\ \hline
t                        & 0.00 & 0.03 & 0.06 & 0.03 & 0.09 & 0.03 & 0.00 & 0.29 & 0.00 & 0.00 & 0.00 & 0.00 & 0.04 & 0.01 & 0.00 & 0.00 & 0.00 & 0.02 & 0.00 & 0.35 & 0.00 & 0.00 & 0.03 & 0.00 & 0.00 & 0.00 & 0.00 & 0.00 & 0.00   \\ \hline
,                        & 0.00 & 0.00 & 0.00 & 0.00 & 0.00 & 0.00 & 0.00 & 1.00 & 0.00 & 0.00 & 0.00 & 0.00 & 0.00 & 0.00 & 0.00 & 0.00 & 0.00 & 0.00 & 0.00 & 0.00 & 0.00 & 0.00 & 0.00 & 0.00 & 0.00 & 0.00 & 0.00 & 0.00 & 0.00   \\ \hline
                         & 0.04 & 0.03 & 0.01 & 0.01 & 0.06 & 0.14 & 0.00 & 0.00 & 0.05 & 0.02 & 0.05 & 0.06 & 0.04 & 0.00 & 0.04 & 0.01 & 0.07 & 0.13 & 0.00 & 0.09 & 0.04 & 0.02 & 0.06 & 0.01 & 0.01 & 0.00 & 0.01 & 0.00 & 0.00   \\ \hline
d                        & 0.00 & 0.05 & 0.10 & 0.04 & 0.10 & 0.00 & 0.00 & 0.58 & 0.01 & 0.00 & 0.01 & 0.00 & 0.02 & 0.01 & 0.00 & 0.01 & 0.00 & 0.04 & 0.00 & 0.00 & 0.00 & 0.00 & 0.04 & 0.00 & 0.00 & 0.00 & 0.00 & 0.00 & 0.00   \\ \hline
g                        & 0.00 & 0.04 & 0.09 & 0.12 & 0.03 & 0.01 & 0.00 & 0.32 & 0.00 & 0.04 & 0.00 & 0.00 & 0.01 & 0.04 & 0.00 & 0.02 & 0.00 & 0.04 & 0.00 & 0.12 & 0.00 & 0.01 & 0.11 & 0.00 & 0.00 & 0.00 & 0.00 & 0.00 & 0.00   \\ \hline
f                        & 0.00 & 0.07 & 0.09 & 0.10 & 0.19 & 0.04 & 0.00 & 0.33 & 0.00 & 0.00 & 0.04 & 0.00 & 0.04 & 0.07 & 0.00 & 0.00 & 0.00 & 0.06 & 0.00 & 0.00 & 0.00 & 0.00 & 0.00 & 0.00 & 0.00 & 0.00 & 0.00 & 0.00 & 0.00   \\ \hline
b                        & 0.00 & 0.10 & 0.21 & 0.13 & 0.11 & 0.00 & 0.00 & 0.02 & 0.00 & 0.00 & 0.00 & 0.02 & 0.13 & 0.13 & 0.00 & 0.03 & 0.00 & 0.10 & 0.00 & 0.00 & 0.00 & 0.00 & 0.00 & 0.00 & 0.00 & 0.00 & 0.00 & 0.00 & 0.00   \\ \hline
l                        & 0.00 & 0.12 & 0.21 & 0.00 & 0.11 & 0.01 & 0.00 & 0.10 & 0.09 & 0.02 & 0.01 & 0.00 & 0.12 & 0.03 & 0.00 & 0.07 & 0.00 & 0.07 & 0.00 & 0.00 & 0.00 & 0.00 & 0.03 & 0.00 & 0.02 & 0.00 & 0.00 & 0.00 & 0.00   \\ \hline
u                        & 0.06 & 0.02 & 0.02 & 0.10 & 0.00 & 0.15 & 0.00 & 0.01 & 0.02 & 0.10 & 0.00 & 0.01 & 0.12 & 0.00 & 0.04 & 0.01 & 0.00 & 0.02 & 0.00 & 0.00 & 0.03 & 0.14 & 0.14 & 0.00 & 0.01 & 0.00 & 0.00 & 0.00 & 0.01   \\ \hline
m                        & 0.02 & 0.07 & 0.25 & 0.00 & 0.09 & 0.00 & 0.01 & 0.20 & 0.00 & 0.00 & 0.00 & 0.05 & 0.00 & 0.05 & 0.00 & 0.00 & 0.00 & 0.20 & 0.00 & 0.00 & 0.00 & 0.00 & 0.05 & 0.00 & 0.00 & 0.00 & 0.00 & 0.00 & 0.00   \\ \hline
y                        & 0.00 & 0.00 & 0.09 & 0.00 & 0.04 & 0.01 & 0.00 & 0.80 & 0.00 & 0.00 & 0.00 & 0.01 & 0.00 & 0.00 & 0.00 & 0.00 & 0.00 & 0.01 & 0.00 & 0.00 & 0.00 & 0.00 & 0.02 & 0.00 & 0.01 & 0.00 & 0.00 & 0.00 & 0.00   \\ \hline
w                        & 0.00 & 0.12 & 0.17 & 0.01 & 0.12 & 0.00 & 0.00 & 0.08 & 0.00 & 0.00 & 0.00 & 0.00 & 0.01 & 0.00 & 0.00 & 0.00 & 0.00 & 0.27 & 0.00 & 0.14 & 0.00 & 0.06 & 0.02 & 0.00 & 0.01 & 0.00 & 0.00 & 0.00 & 0.00   \\ \hline
a                        & 0.01 & 0.02 & 0.01 & 0.11 & 0.00 & 0.10 & 0.00 & 0.06 & 0.07 & 0.03 & 0.01 & 0.03 & 0.06 & 0.02 & 0.01 & 0.04 & 0.02 & 0.00 & 0.00 & 0.00 & 0.04 & 0.26 & 0.08 & 0.01 & 0.02 & 0.00 & 0.00 & 0.00 & 0.00   \\ \hline
.                        & 0.00 & 0.00 & 0.00 & 0.00 & 0.00 & 0.00 & 0.00 & 1.00 & 0.00 & 0.00 & 0.00 & 0.00 & 0.00 & 0.00 & 0.00 & 0.00 & 0.00 & 0.00 & 0.00 & 0.00 & 0.00 & 0.00 & 0.00 & 0.00 & 0.00 & 0.00 & 0.00 & 0.00 & 0.00   \\ \hline
h                        & 0.00 & 0.13 & 0.48 & 0.00 & 0.08 & 0.04 & 0.00 & 0.09 & 0.00 & 0.00 & 0.00 & 0.00 & 0.00 & 0.03 & 0.00 & 0.00 & 0.00 & 0.15 & 0.00 & 0.00 & 0.00 & 0.00 & 0.00 & 0.00 & 0.00 & 0.00 & 0.00 & 0.00 & 0.00   \\ \hline
c                        & 0.00 & 0.03 & 0.11 & 0.00 & 0.19 & 0.07 & 0.00 & 0.02 & 0.00 & 0.00 & 0.00 & 0.00 & 0.05 & 0.03 & 0.00 & 0.00 & 0.00 & 0.16 & 0.00 & 0.25 & 0.01 & 0.00 & 0.00 & 0.00 & 0.09 & 0.00 & 0.00 & 0.00 & 0.00   \\ \hline
n                        & 0.00 & 0.02 & 0.06 & 0.05 & 0.04 & 0.06 & 0.00 & 0.25 & 0.29 & 0.12 & 0.00 & 0.00 & 0.00 & 0.00 & 0.00 & 0.01 & 0.00 & 0.03 & 0.00 & 0.00 & 0.03 & 0.01 & 0.02 & 0.00 & 0.01 & 0.00 & 0.00 & 0.00 & 0.00   \\ \hline
s                        & 0.01 & 0.04 & 0.08 & 0.00 & 0.08 & 0.11 & 0.00 & 0.49 & 0.00 & 0.00 & 0.00 & 0.00 & 0.01 & 0.02 & 0.01 & 0.01 & 0.01 & 0.03 & 0.00 & 0.05 & 0.01 & 0.00 & 0.02 & 0.00 & 0.00 & 0.02 & 0.00 & 0.00 & 0.00   \\ \hline
v                        & 0.00 & 0.13 & 0.73 & 0.00 & 0.03 & 0.00 & 0.00 & 0.00 & 0.00 & 0.00 & 0.00 & 0.00 & 0.00 & 0.00 & 0.00 & 0.00 & 0.00 & 0.10 & 0.00 & 0.00 & 0.00 & 0.00 & 0.00 & 0.00 & 0.00 & 0.00 & 0.00 & 0.00 & 0.00   \\ \hline
k                        & 0.00 & 0.15 & 0.40 & 0.00 & 0.00 & 0.00 & 0.00 & 0.33 & 0.00 & 0.00 & 0.00 & 0.00 & 0.04 & 0.00 & 0.00 & 0.00 & 0.02 & 0.02 & 0.00 & 0.00 & 0.00 & 0.02 & 0.02 & 0.00 & 0.00 & 0.00 & 0.00 & 0.00 & 0.00   \\ \hline
q                        & 0.00 & 0.00 & 0.00 & 0.00 & 0.00 & 0.00 & 0.00 & 0.00 & 0.00 & 0.00 & 0.00 & 0.00 & 0.00 & 1.00 & 0.00 & 0.00 & 0.00 & 0.00 & 0.00 & 0.00 & 0.00 & 0.00 & 0.00 & 0.00 & 0.00 & 0.00 & 0.00 & 0.00 & 0.00   \\ \hline
j                        & 0.00 & 0.00 & 0.87 & 0.00 & 0.13 & 0.00 & 0.00 & 0.00 & 0.00 & 0.00 & 0.00 & 0.00 & 0.00 & 0.00 & 0.00 & 0.00 & 0.00 & 0.00 & 0.00 & 0.00 & 0.00 & 0.00 & 0.00 & 0.00 & 0.00 & 0.00 & 0.00 & 0.00 & 0.00   \\ \hline
x                        & 1.00 & 0.00 & 0.00 & 0.00 & 0.00 & 0.00 & 0.00 & 0.00 & 0.00 & 0.00 & 0.00 & 0.00 & 0.00 & 0.00 & 0.00 & 0.00 & 0.00 & 0.00 & 0.00 & 0.00 & 0.00 & 0.00 & 0.00 & 0.00 & 0.00 & 0.00 & 0.00 & 0.00 & 0.00   \\ \hline
z                        & 0.00 & 0.17 & 0.50 & 0.00 & 0.00 & 0.00 & 0.00 & 0.00 & 0.00 & 0.00 & 0.00 & 0.00 & 0.00 & 0.00 & 0.00 & 0.17 & 0.00 & 0.00 & 0.00 & 0.00 & 0.00 & 0.00 & 0.00 & 0.00 & 0.00 & 0.00 & 0.00 & 0.00 & 0.17   \\ \hline
\end{tabular}
}
\end{center}
\end{table}
\end{landscape}
\pagestyle{plain}


    
    \subsection{Příklad 3}\label{pux159uxedklad-3}

    Při hledání stacionárního rozdělení vycházíme z rovnosti
\(\pi = \pi P\). Dále že pokud platí \(\vec{v} P = \lambda \vec{v}\),
potom také \((\alpha \vec{v}) P = \lambda (\alpha \vec{v})\). Z
dokumnetace scipy metoda eig vrací jako druhý argument: \emph{The
normalized left eigenvector corresponding to the eigenvalue w{[}i{]} is
the column vl{[}:,i{]}. Only returned if left=True}.

    \begin{Verbatim}[commandchars=\\\{\}]
{\color{incolor}In [{\color{incolor}7}]:} \PY{c+c1}{\PYZsh{} Solve an ordinary or generalized eigenvalue problem of a square matrix.}
        \PY{n}{v} \PY{o}{=} \PY{n}{scipy}\PY{o}{.}\PY{n}{linalg}\PY{o}{.}\PY{n}{eig}\PY{p}{(}\PY{n}{P}\PY{o}{.}\PY{n}{astype}\PY{p}{(}\PY{n}{np}\PY{o}{.}\PY{n}{float}\PY{p}{)}\PY{p}{,}\PY{n}{left}\PY{o}{=}\PY{k+kc}{True}\PY{p}{,}\PY{n}{right}\PY{o}{=}\PY{k+kc}{False}\PY{p}{)}
        \PY{c+c1}{\PYZsh{} The eigenvalues, each repeated according to its multiplicity.}
        \PY{n}{v} \PY{o}{=} \PY{n}{v}\PY{p}{[}\PY{l+m+mi}{1}\PY{p}{]}
        \PY{c+c1}{\PYZsh{} get first column}
        \PY{n}{v} \PY{o}{=} \PY{n}{v}\PY{p}{[}\PY{p}{:}\PY{p}{,}\PY{l+m+mi}{0}\PY{p}{]}
        \PY{n}{stacionary\PYZus{}distribution} \PY{o}{=} \PY{n}{v} \PY{o}{/} \PY{n+nb}{sum}\PY{p}{(}\PY{n}{v}\PY{p}{)}
        \PY{n+nb}{print}\PY{p}{(}\PY{l+s+s2}{\PYZdq{}}\PY{l+s+s2}{Stacionary distribution}\PY{l+s+s2}{\PYZdq{}}\PY{p}{)}
        \PY{n+nb}{print}\PY{p}{(}\PY{n}{stacionary\PYZus{}distribution}\PY{o}{.}\PY{n}{real}\PY{p}{)}
        \PY{n+nb}{print}\PY{p}{(}\PY{l+s+s2}{\PYZdq{}}\PY{l+s+se}{\PYZbs{}n}\PY{l+s+s2}{Check sum: }\PY{l+s+s2}{\PYZdq{}}\PY{p}{,} \PY{n}{stacionary\PYZus{}distribution}\PY{o}{.}\PY{n}{sum}\PY{p}{(}\PY{p}{)}\PY{o}{.}\PY{n}{real}\PY{p}{)}
        \PY{n+nb}{print}\PY{p}{(}\PY{p}{)}
        \PY{n}{stacionary\PYZus{}distribution\PYZus{}round} \PY{o}{=} \PY{p}{(}\PY{n}{v} \PY{o}{/} \PY{n+nb}{sum}\PY{p}{(}\PY{n}{v}\PY{p}{)}\PY{p}{)}\PY{o}{.}\PY{n}{round}\PY{p}{(}\PY{l+m+mi}{4}\PY{p}{)}\PY{o}{.}\PY{n}{real}
        \PY{n+nb}{print}\PY{p}{(}\PY{n}{stacionary\PYZus{}distribution\PYZus{}round}\PY{p}{)}
        \PY{n+nb}{print}\PY{p}{(}\PY{p}{)}
        \PY{n+nb}{print}\PY{p}{(}\PY{l+s+s2}{\PYZdq{}}\PY{l+s+se}{\PYZbs{}n}\PY{l+s+s2}{Check sum after round: }\PY{l+s+s2}{\PYZdq{}}\PY{p}{,} \PY{n}{stacionary\PYZus{}distribution\PYZus{}round}\PY{o}{.}\PY{n}{sum}\PY{p}{(}\PY{p}{)}\PY{p}{)}
\end{Verbatim}

    \begin{Verbatim}[commandchars=\\\{\}]
Stacionary distribution
[1.39846205e-02 4.59924066e-02 1.01994761e-01 5.44738342e-02
 6.18536309e-02 6.73979107e-02 3.07747283e-04 1.85601668e-01
 4.69262579e-02 2.20088141e-02 2.12347538e-02 1.50814659e-02
 3.85932560e-02 2.46153263e-02 1.69257852e-02 1.30767711e-02
 1.81592446e-02 6.77060272e-02 1.53877335e-04 5.18785635e-02
 1.56952306e-02 5.46203074e-02 4.47920949e-02 4.61583354e-03
 8.00092027e-03 9.23461875e-04 2.30847846e-03 1.53838252e-04
 9.23112969e-04]

Check sum:  0.9999999999999999

[0.014  0.046  0.102  0.0545 0.0619 0.0674 0.0003 0.1856 0.0469 0.022
 0.0212 0.0151 0.0386 0.0246 0.0169 0.0131 0.0182 0.0677 0.0002 0.0519
 0.0157 0.0546 0.0448 0.0046 0.008  0.0009 0.0023 0.0002 0.0009]


Check sum after round:  1.0001

    \end{Verbatim}

\pagebreak

    \subsection{Příklad 4}\label{pux159uxedklad-4}

    \begin{itemize}
\tightlist
\item
  Na následujícím grafu je znázorněno srovnání empirických četností
  výskytu znaků s teoretickými četnostmi, které vycházejí ze
  stacionárního rozdělení.
\item
  Je dobře vidět, že jsou obě sady četností téměř identické, tudíž
  předpokládáme, že ani nebude možné zamítnout jejich shodu ve prospěch
  alternativy, tedy že jsou nezávislé. K tomuto využijeme testu dobré
  shody.
\end{itemize}

    \begin{Verbatim}[commandchars=\\\{\}]
{\color{incolor}In [{\color{incolor}8}]:} \PY{n}{y\PYZus{}values} \PY{o}{=} \PY{p}{[}\PY{p}{]}
        \PY{k}{for} \PY{n}{l} \PY{o+ow}{in} \PY{n}{xletters}\PY{p}{:}
            \PY{n}{y\PYZus{}values}\PY{o}{.}\PY{n}{append}\PY{p}{(}\PY{n}{stacionary\PYZus{}distribution}\PY{o}{.}\PY{n}{real}\PY{p}{[}\PY{n}{letter\PYZus{}to\PYZus{}key}\PY{p}{[}\PY{n}{l}\PY{p}{]}\PY{p}{]}\PY{p}{)}
\end{Verbatim}

    \begin{Verbatim}[commandchars=\\\{\}]
{\color{incolor}In [{\color{incolor}9}]:} \PY{n}{fig}\PY{p}{,} \PY{n}{ax} \PY{o}{=} \PY{n}{plt}\PY{o}{.}\PY{n}{subplots}\PY{p}{(}\PY{l+m+mi}{1}\PY{p}{,} \PY{l+m+mi}{1}\PY{p}{,} \PY{n}{figsize}\PY{o}{=}\PY{p}{(}\PY{l+m+mi}{15}\PY{p}{,} \PY{l+m+mi}{5}\PY{p}{)}\PY{p}{)}
        \PY{n}{ax}\PY{o}{.}\PY{n}{bar}\PY{p}{(}\PY{n}{xletters}\PY{p}{,} \PY{n}{xletter\PYZus{}cnt}\PY{p}{,} \PY{n}{fc}\PY{o}{=}\PY{p}{(}\PY{l+m+mi}{0}\PY{p}{,} \PY{l+m+mi}{1}\PY{p}{,} \PY{l+m+mf}{0.48}\PY{p}{,} \PY{l+m+mi}{1}\PY{p}{)}\PY{p}{,} \PY{n}{lw}\PY{o}{=}\PY{l+m+mi}{3}\PY{p}{)}
        \PY{n}{ax}\PY{o}{.}\PY{n}{bar}\PY{p}{(}\PY{n}{xletters}\PY{p}{,} \PY{n}{np}\PY{o}{.}\PY{n}{array}\PY{p}{(}\PY{n}{y\PYZus{}values}\PY{p}{)}\PY{o}{*}\PY{n+nb}{len}\PY{p}{(}\PY{n}{xfile}\PY{p}{)}\PY{p}{,} \PY{n}{fc}\PY{o}{=}\PY{p}{(}\PY{l+m+mi}{1}\PY{p}{,} \PY{l+m+mi}{0}\PY{p}{,} \PY{l+m+mf}{0.52}\PY{p}{,} \PY{l+m+mf}{0.5}\PY{p}{)}\PY{p}{,} \PY{n}{lw}\PY{o}{=}\PY{l+m+mi}{3}\PY{p}{)}
        \PY{n}{ax}\PY{o}{.}\PY{n}{bar}\PY{p}{(}\PY{p}{[}\PY{l+m+mi}{1}\PY{p}{]}\PY{p}{,} \PY{p}{[}\PY{l+m+mi}{0}\PY{p}{]}\PY{p}{,} \PY{n}{color}\PY{o}{=}\PY{p}{(}\PY{l+m+mi}{0}\PY{p}{,} \PY{l+m+mi}{0}\PY{p}{,} \PY{l+m+mf}{0.249}\PY{p}{,} \PY{l+m+mf}{0.5}\PY{p}{)}\PY{p}{)}
        \PY{n}{ax}\PY{o}{.}\PY{n}{set\PYZus{}xlabel}\PY{p}{(}\PY{l+s+s1}{\PYZsq{}}\PY{l+s+s1}{znak}\PY{l+s+s1}{\PYZsq{}}\PY{p}{,} \PY{n}{fontsize}\PY{o}{=}\PY{l+m+mi}{16}\PY{p}{)}
        \PY{n}{ax}\PY{o}{.}\PY{n}{set\PYZus{}ylabel}\PY{p}{(}\PY{l+s+s1}{\PYZsq{}}\PY{l+s+s1}{počet}\PY{l+s+s1}{\PYZsq{}}\PY{p}{,} \PY{n}{fontsize}\PY{o}{=}\PY{l+m+mi}{16}\PY{p}{)}
        \PY{n}{ax}\PY{o}{.}\PY{n}{set\PYZus{}title}\PY{p}{(}\PY{l+s+s1}{\PYZsq{}}\PY{l+s+s1}{Empirické a stacionární četnosti}\PY{l+s+s1}{\PYZsq{}}\PY{p}{,} \PY{n}{fontsize}\PY{o}{=}\PY{l+m+mi}{22}\PY{p}{)}
        \PY{n}{plt}\PY{o}{.}\PY{n}{legend}\PY{p}{(}\PY{p}{[}\PY{l+s+s1}{\PYZsq{}}\PY{l+s+s1}{Výběrové četnosti}\PY{l+s+s1}{\PYZsq{}}\PY{p}{,} \PY{l+s+s1}{\PYZsq{}}\PY{l+s+s1}{Stacionární četnosti}\PY{l+s+s1}{\PYZsq{}}\PY{p}{,} \PY{l+s+s1}{\PYZsq{}}\PY{l+s+s1}{Průnik}\PY{l+s+s1}{\PYZsq{}}\PY{p}{]}\PY{p}{,} \PY{n}{fontsize}\PY{o}{=}\PY{l+m+mi}{16}\PY{p}{)}
        \PY{k}{for} \PY{n}{tick} \PY{o+ow}{in} \PY{n}{ax}\PY{o}{.}\PY{n}{xaxis}\PY{o}{.}\PY{n}{get\PYZus{}major\PYZus{}ticks}\PY{p}{(}\PY{p}{)}\PY{p}{:}
            \PY{n}{tick}\PY{o}{.}\PY{n}{label}\PY{o}{.}\PY{n}{set\PYZus{}fontsize}\PY{p}{(}\PY{l+m+mi}{14}\PY{p}{)}
        \PY{k}{for} \PY{n}{tick} \PY{o+ow}{in} \PY{n}{ax}\PY{o}{.}\PY{n}{yaxis}\PY{o}{.}\PY{n}{get\PYZus{}major\PYZus{}ticks}\PY{p}{(}\PY{p}{)}\PY{p}{:}
            \PY{n}{tick}\PY{o}{.}\PY{n}{label}\PY{o}{.}\PY{n}{set\PYZus{}fontsize}\PY{p}{(}\PY{l+m+mi}{14}\PY{p}{)}
        \PY{n}{plt}\PY{o}{.}\PY{n}{show}\PY{p}{(}\PY{p}{)}
\end{Verbatim}

    \begin{center}
    \adjustimage{max size={0.9\linewidth}{0.9\paperheight}}{output_20_0.png}
    \end{center}
    { \hspace*{\fill} \\}
    \pagebreak
    
    \begin{Verbatim}[commandchars=\\\{\}]
{\color{incolor}In [{\color{incolor}10}]:} \PY{n}{stationary\PYZus{}counts} \PY{o}{=} \PY{n}{np}\PY{o}{.}\PY{n}{array}\PY{p}{(}\PY{n}{y\PYZus{}values}\PY{p}{)}\PY{o}{*}\PY{n+nb}{len}\PY{p}{(}\PY{n}{xfile}\PY{p}{)}
\end{Verbatim}

    \begin{Verbatim}[commandchars=\\\{\}]
{\color{incolor}In [{\color{incolor}11}]:} \PY{k}{def} \PY{n+nf}{merge\PYZus{}bins}\PY{p}{(}\PY{n}{counts}\PY{p}{)}\PY{p}{:}
             \PY{n}{flag} \PY{o}{=} \PY{k+kc}{True}
             \PY{k}{while}\PY{p}{(}\PY{n}{flag}\PY{p}{)}\PY{p}{:}
                 \PY{n}{flag} \PY{o}{=} \PY{k+kc}{False}
                 \PY{n}{merged\PYZus{}arr} \PY{o}{=} \PY{p}{[}\PY{p}{]}
                 \PY{n}{waiting} \PY{o}{=} \PY{l+m+mi}{0}
                 \PY{k}{for} \PY{n}{val} \PY{o+ow}{in} \PY{n}{counts}\PY{p}{:}
                     \PY{k}{if} \PY{n}{val}\PY{o}{+}\PY{n}{waiting} \PY{o}{\PYZgt{}} \PY{l+m+mi}{4}\PY{p}{:}
                         \PY{n}{merged\PYZus{}arr}\PY{o}{.}\PY{n}{append}\PY{p}{(}\PY{n}{val}\PY{o}{+}\PY{n}{waiting}\PY{p}{)}
                         \PY{n}{waiting} \PY{o}{=} \PY{l+m+mi}{0}
                     \PY{k}{else}\PY{p}{:}
                         \PY{n}{flag} \PY{o}{=} \PY{k+kc}{True}
                         \PY{n}{waiting} \PY{o}{+}\PY{o}{=} \PY{n}{val}\PY{p}{;}
                 \PY{k}{if} \PY{n}{waiting} \PY{o}{\PYZgt{}} \PY{l+m+mi}{0}\PY{p}{:}
                     \PY{n}{merged\PYZus{}arr}\PY{p}{[}\PY{o}{\PYZhy{}}\PY{l+m+mi}{1}\PY{p}{]} \PY{o}{+}\PY{o}{=} \PY{n}{waiting}
                 \PY{n}{counts} \PY{o}{=} \PY{n}{merged\PYZus{}arr}
             \PY{k}{return} \PY{n}{np}\PY{o}{.}\PY{n}{array}\PY{p}{(}\PY{n}{counts}\PY{p}{)}
\end{Verbatim}

    \begin{Verbatim}[commandchars=\\\{\}]
{\color{incolor}In [{\color{incolor}12}]:} \PY{n}{merged\PYZus{}stationary\PYZus{}counts} \PY{o}{=} \PY{n}{merge\PYZus{}bins}\PY{p}{(}\PY{n}{stationary\PYZus{}counts}\PY{p}{)}
         \PY{n}{merged\PYZus{}xletter\PYZus{}cnt} \PY{o}{=} \PY{n}{merge\PYZus{}bins}\PY{p}{(}\PY{n}{xletter\PYZus{}cnt}\PY{p}{)}
         \PY{n+nb}{print}\PY{p}{(}\PY{l+s+s2}{\PYZdq{}}\PY{l+s+s2}{Četnosti mají shodnou délku po spojení binů?}\PY{l+s+s2}{\PYZdq{}}\PY{p}{,} 
         \PY{l+s+s2}{\PYZdq{}}\PY{l+s+s2}{Ano}\PY{l+s+s2}{\PYZdq{}} \PY{k}{if} \PY{n+nb}{len}\PY{p}{(}\PY{n}{merged\PYZus{}stationary\PYZus{}counts}\PY{p}{)} \PY{o}{==} \PY{n+nb}{len}\PY{p}{(}\PY{n}{merged\PYZus{}xletter\PYZus{}cnt}\PY{p}{)} \PY{k}{else} \PY{l+s+s2}{\PYZdq{}}\PY{l+s+s2}{Ne}\PY{l+s+s2}{\PYZdq{}}\PY{p}{)}
\end{Verbatim}

    \begin{Verbatim}[commandchars=\\\{\}]
Četnosti mají shodnou délku po spojení binů? Ano

    \end{Verbatim}

    \subsubsection{Test dobré shody}\label{test-dobruxe9-shody}

\begin{itemize}
\tightlist
\item
  \(H_0\) pro nás představuje dobrou shodu empirických a stacionárního
  rozdělení,
\item
  \(H_A\) představuje jejich nezávislost.
\end{itemize}

    \begin{Verbatim}[commandchars=\\\{\}]
{\color{incolor}In [{\color{incolor}13}]:} \PY{n}{chi}\PY{p}{,} \PY{n}{p} \PY{o}{=} \PY{n}{stats}\PY{o}{.}\PY{n}{chisquare}\PY{p}{(}\PY{n}{merged\PYZus{}xletter\PYZus{}cnt}\PY{p}{,} \PY{n}{merged\PYZus{}stationary\PYZus{}counts}\PY{p}{)}
         \PY{n}{crit} \PY{o}{=} \PY{n}{stats}\PY{o}{.}\PY{n}{chi2}\PY{o}{.}\PY{n}{isf}\PY{p}{(}\PY{l+m+mf}{0.05}\PY{p}{,} \PY{n+nb}{len}\PY{p}{(}\PY{n}{merged\PYZus{}stationary\PYZus{}counts}\PY{p}{)} \PY{o}{\PYZhy{}} \PY{l+m+mi}{1}\PY{p}{)}
         \PY{n+nb}{print}\PY{p}{(}\PY{l+s+s2}{\PYZdq{}}\PY{l+s+s2}{Hodnota testové statistiky: }\PY{l+s+s2}{\PYZdq{}}\PY{p}{,} \PY{n}{chi}\PY{p}{)}
         \PY{n+nb}{print}\PY{p}{(}\PY{l+s+s2}{\PYZdq{}}\PY{l+s+s2}{Kritická hodnota: }\PY{l+s+s2}{\PYZdq{}}\PY{p}{,} \PY{n}{crit}\PY{p}{)}
         \PY{n+nb}{print}\PY{p}{(}\PY{l+s+s2}{\PYZdq{}}\PY{l+s+s2}{p\PYZhy{}hodnota:}\PY{l+s+s2}{\PYZdq{}}\PY{p}{,} \PY{n}{p}\PY{p}{)}
         \PY{n+nb}{print}\PY{p}{(}\PY{l+s+s2}{\PYZdq{}}\PY{l+s+s2}{\PYZhy{}}\PY{l+s+s2}{\PYZdq{}}\PY{o}{*}\PY{l+m+mi}{20} \PY{o}{+} \PY{l+s+s2}{\PYZdq{}}\PY{l+s+se}{\PYZbs{}n}\PY{l+s+s2}{Zamítáme: }\PY{l+s+s2}{\PYZdq{}}\PY{p}{,} \PY{l+s+s2}{\PYZdq{}}\PY{l+s+s2}{Ano}\PY{l+s+s2}{\PYZdq{}} \PY{k}{if} \PY{n}{chi} \PY{o}{\PYZgt{}}\PY{o}{=} \PY{n}{crit} \PY{k}{else} \PY{l+s+s2}{\PYZdq{}}\PY{l+s+s2}{Ne}\PY{l+s+s2}{\PYZdq{}}\PY{p}{)}
\end{Verbatim}

    \begin{Verbatim}[commandchars=\\\{\}]
Hodnota testové statistiky:  0.014022647231326082
Kritická hodnota:  37.65248413348277
p-hodnota: 1.0
--------------------
Zamítáme:  Ne

    \end{Verbatim}

    \begin{itemize}
\tightlist
\item
  Na hladině 5\% nezámítáme hypotézu \(H_0\), tedy \(H_A\) je
  statisticky nevýznamné a může nastat chyba 2. druhu o které nic
  nevíme. Ale z vysoké p-hodnoty můžeme s téměř sto procentní jistotou
  říct, že chyba 2. druhu nenastane a platí hypotéza \(H_0\), jak jsme
  předpokládali. Tedy empirické četnosti a vektor stacionárního
  rozdělení mají shodné rozdělení.
\end{itemize}


    % Add a bibliography block to the postdoc
    
    
    
    \end{document}
